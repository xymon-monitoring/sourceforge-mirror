\chapter{Managing releases and branchy development}
\label{chap:branch}

Mercurial provides several mechanisms for you to manage a project that
is making progress on multiple fronts at once.  To understand these
mechanisms, let's first take a brief look at a fairly normal software
project structure.

Many software projects issue periodic ``major'' releases that contain
substantial new features.  In parallel, they may issue ``minor''
releases.  These are usually identical to the major releases off which
they're based, but with a few bugs fixed.

In this chapter, we'll start by talking about how to keep records of
project milestones such as releases.  We'll then continue on to talk
about the flow of work between different phases of a project, and how
Mercurial can help you to isolate and manage this work.

\section{Giving a persistent name to a revision}

Once you decide that you'd like to call a particular revision a
``release'', it's a good idea to record the identity of that revision.
This will let you reproduce that release at a later date, for whatever
purpose you might need at the time (reproducing a bug, porting to a
new platform, etc).
\interaction{tag.init}

Mercurial lets you give a permanent name to any revision using the
\hgcmd{tag} command.  Not surprisingly, these names are called
``tags''.
\interaction{tag.tag}

A tag is nothing more than a ``symbolic name'' for a revision.  Tags
exist purely for your convenience, so that you have a handy permanent
way to refer to a revision; Mercurial doesn't interpret the tag names
you use in any way.  Neither does Mercurial place any restrictions on
the name of a tag, beyond a few that are necessary to ensure that a
tag can be parsed unambiguously.  A tag name cannot contain any of the
following characters:
\begin{itemize}
\item Colon (ASCII 58, ``\texttt{:}'')
\item Carriage return (ASCII 13, ``\Verb+\r+'')
\item Newline (ASCII 10, ``\Verb+\n+'')
\end{itemize}

You can use the \hgcmd{tags} command to display the tags present in
your repository.  In the output, each tagged revision is identified
first by its name, then by revision number, and finally by the unique
hash of the revision.  
\interaction{tag.tags}
Notice that \texttt{tip} is listed in the output of \hgcmd{tags}.  The
\texttt{tip} tag is a special ``floating'' tag, which always
identifies the newest revision in the repository.

In the output of the \hgcmd{tags} command, tags are listed in reverse
order, by revision number.  This usually means that recent tags are
listed before older tags.  It also means that \texttt{tip} is always
going to be the first tag listed in the output of \hgcmd{tags}.

When you run \hgcmd{log}, if it displays a revision that has tags
associated with it, it will print those tags.
\interaction{tag.log}

Any time you need to provide a revision~ID to a Mercurial command, the
command will accept a tag name in its place.  Internally, Mercurial
will translate your tag name into the corresponding revision~ID, then
use that.
\interaction{tag.log.v1.0}

There's no limit on the number of tags you can have in a repository,
or on the number of tags that a single revision can have.  As a
practical matter, it's not a great idea to have ``too many'' (a number
which will vary from project to project), simply because tags are
supposed to help you to find revisions.  If you have lots of tags, the
ease of using them to identify revisions diminishes rapidly.

For example, if your project has milestones as frequent as every few
days, it's perfectly reasonable to tag each one of those.  But if you
have a continuous build system that makes sure every revision can be
built cleanly, you'd be introducing a lot of noise if you were to tag
every clean build.  Instead, you could tag failed builds (on the
assumption that they're rare!), or simply not use tags to track
buildability.

If you want to remove a tag that you no longer want, use
\hgcmdargs{tag}{--remove}.  
\interaction{tag.remove}
You can also modify a tag at any time, so that it identifies a
different revision, by simply issuing a new \hgcmd{tag} command.
You'll have to use the \hgopt{tag}{-f} option to tell Mercurial that
you \emph{really} want to update the tag.
\interaction{tag.replace}
There will still be a permanent record of the previous identity of the
tag, but Mercurial will no longer use it.  There's thus no penalty to
tagging the wrong revision; all you have to do is turn around and tag
the correct revision once you discover your error.

Mercurial stores tags in a normal revision-controlled file in your
repository.  If you've created any tags, you'll find them in a file
named \sfilename{.hgtags}.  When you run the \hgcmd{tag} command,
Mercurial modifies this file, then automatically commits the change to
it.  This means that every time you run \hgcmd{tag}, you'll see a
corresponding changeset in the output of \hgcmd{log}.
\interaction{tag.tip}

\section{Handling tag conflicts during a merge}

You won't often need to care about the \sfilename{.hgtags} file, but
it sometimes makes its presence known during a merge.  The format of
the file is simple: it consists of a series of lines.  Each line
starts with a changeset hash, followed by a space, followed by the
name of a tag.

If you're resolving a conflict in the \sfilename{.hgtags} file during
a merge, there's one twist to modifying the \sfilename{.hgtags} file:
when Mercurial is parsing the tags in a repository, it \emph{never}
reads the working copy of the \sfilename{.hgtags} file.  Instead, it
reads the \emph{most recently committed} revision of the file.

An unfortunate consequence of this design is that you can't actually
verify that your merged \sfilename{.hgtags} file is correct until
\emph{after} you've committed a change.  So if you find yourself
resolving a conflict on \sfilename{.hgtags} during a merge, be sure to
run \hgcmd{tags} after you commit.  If it finds an error in the
\sfilename{.hgtags} file, it will report the location of the error,
which you can then fix and commit.  You should then run \hgcmd{tags}
again, just to be sure that your fix is correct.

\section{Tags and cloning}

You may have noticed that the \hgcmd{clone} command has a
\hgopt{clone}{-r} option that lets you clone an exact copy of the
repository as of a particular changeset.  The new clone will not
contain any project history that comes after the revision you
specified.  This has an interaction with tags that can surprise the
unwary.

Recall that a tag is stored as a revision to the \sfilename{.hgtags}
file, so that when you create a tag, the changeset in which it's
recorded necessarily refers to an older changeset.  When you run
\hgcmdargs{clone}{-r foo} to clone a repository as of tag
\texttt{foo}, the new clone \emph{will not contain the history that
  created the tag} that you used to clone the repository.  The result
is that you'll get exactly the right subset of the project's history
in the new repository, but \emph{not} the tag you might have expected.

\section{When permanent tags are too much}

Since Mercurial's tags are revision controlled and carried around with
a project's history, everyone you work with will see the tags you
create.  But giving names to revisions has uses beyond simply noting
that revision \texttt{4237e45506ee} is really \texttt{v2.0.2}.  If
you're trying to track down a subtle bug, you might want a tag to
remind you of something like ``Anne saw the symptoms with this
revision''.

For cases like this, what you might want to use are \emph{local} tags.
You can create a local tag with the \hgopt{tag}{-l} option to the
\hgcmd{tag} command.  This will store the tag in a file called
\sfilename{.hg/localtags}.  Unlike \sfilename{.hgtags},
\sfilename{.hg/localtags} is not revision controlled.  Any tags you
create using \hgopt{tag}{-l} remain strictly local to the repository
you're currently working in.

\section{The flow of changes---big picture vs. little}

To return to the outline I sketched at the beginning of a chapter,
let's think about a project that has multiple concurrent pieces of
work under development at once.

There might be a push for a new ``main'' release; a new minor bugfix
release to the last main release; and an unexpected ``hot fix'' to an
old release that is now in maintenance mode.

The usual way people refer to these different concurrent directions of
development is as ``branches''.  However, we've already seen numerous
times that Mercurial treats \emph{all of history} as a series of
branches and merges.  Really, what we have here is two ideas that are
peripherally related, but which happen to share a name.
\begin{itemize}
\item ``Big picture'' branches represent the sweep of a project's
  evolution; people give them names, and talk about them in
  conversation.
\item ``Little picture'' branches are artefacts of the day-to-day
  activity of developing and merging changes.  They expose the
  narrative of how the code was developed.
\end{itemize}

\section{Managing big-picture branches in repositories}

The easiest way to isolate a ``big picture'' branch in Mercurial is in
a dedicated repository.  If you have an existing shared
repository---let's call it \texttt{myproject}---that reaches a ``1.0''
milestone, you can start to prepare for future maintenance releases on
top of version~1.0 by tagging the revision from which you prepared
the~1.0 release.
\interaction{branch-repo.tag}
You can then clone a new shared \texttt{myproject-1.0.1} repository as
of that tag.
\interaction{branch-repo.clone}

Afterwards, if someone needs to work on a bug fix that ought to go
into an upcoming~1.0.1 minor release, they clone the
\texttt{myproject-1.0.1} repository, make their changes, and push them
back.
\interaction{branch-repo.bugfix}
Meanwhile, development for the next major release can continue,
isolated and unabated, in the \texttt{myproject} repository.
\interaction{branch-repo.new}

\section{Don't repeat yourself: merging across branches}

In many cases, if you have a bug to fix on a maintenance branch, the
chances are good that the bug exists on your project's main branch
(and possibly other maintenance branches, too).  It's a rare developer
who wants to fix the same bug multiple times, so let's look at a few
ways that Mercurial can help you to manage these bugfixes without
duplicating your work.

In the simplest instance, all you need to do is pull changes from your
maintenance branch into your local clone of the target branch.
\interaction{branch-repo.pull}
You'll then need to merge the heads of the two branches, and push back
to the main branch.
\interaction{branch-repo.merge}

\section{Naming branches within one repository}

In most instances, isolating branches in repositories is the right
approach.  Its simplicity makes it easy to understand; and so it's
hard to make mistakes.  There's a one-to-one relationship between
branches you're working in and directories on your system.  This lets
you use normal (non-Mercurial-aware) tools to work on files within a
branch/repository.

If you're more in the ``power user'' category (\emph{and} your
collaborators are too), there is an alternative way of handling
branches that you can consider.  I've already mentioned the
human-level distinction between ``small picture'' and ``big picture''
branches.  While Mercurial works with multiple ``small picture''
branches in a repository all the time (for example after you pull
changes in, but before you merge them), it can \emph{also} work with
multiple ``big picture'' branches.

The key to working this way is that Mercurial lets you assign a
persistent \emph{name} to a branch.  There always exists a branch
named \texttt{default}.  Even before you start naming branches
yourself, you can find traces of the \texttt{default} branch if you
look for them.

As an example, when you run the \hgcmd{commit} command, and it pops up
your editor so that you can enter a commit message, look for a line
that contains the text ``\texttt{HG: branch default}'' at the bottom.
This is telling you that your commit will occur on the branch named
\texttt{default}.

To start working with named branches, use the \hgcmd{branches}
command.  This command lists the named branches already present in
your repository, telling you which changeset is the tip of each.
\interaction{branch-named.branches}
Since you haven't created any named branches yet, the only one that
exists is \texttt{default}.

To find out what the ``current'' branch is, run the \hgcmd{branch}
command, giving it no arguments.  This tells you what branch the
parent of the current changeset is on.
\interaction{branch-named.branch}

To create a new branch, run the \hgcmd{branch} command again.  This
time, give it one argument: the name of the branch you want to create.
\interaction{branch-named.create}

After you've created a branch, you might wonder what effect the
\hgcmd{branch} command has had.  What do the \hgcmd{status} and
\hgcmd{tip} commands report?
\interaction{branch-named.status}
Nothing has changed in the working directory, and there's been no new
history created.  As this suggests, running the \hgcmd{branch} command
has no permanent effect; it only tells Mercurial what branch name to
use the \emph{next} time you commit a changeset.

When you commit a change, Mercurial records the name of the branch on
which you committed.  Once you've switched from the \texttt{default}
branch to another and committed, you'll see the name of the new branch
show up in the output of \hgcmd{log}, \hgcmd{tip}, and other commands
that display the same kind of output.
\interaction{branch-named.commit}
The \hgcmd{log}-like commands will print the branch name of every
changeset that's not on the \texttt{default} branch.  As a result, if
you never use named branches, you'll never see this information.

Once you've named a branch and committed a change with that name,
every subsequent commit that descends from that change will inherit
the same branch name.  You can change the name of a branch at any
time, using the \hgcmd{branch} command.  
\interaction{branch-named.rebranch}
In practice, this is something you won't do very often, as branch
names tend to have fairly long lifetimes.  (This isn't a rule, just an
observation.)

\section{Dealing with multiple named branches in a repository}

If you have more than one named branch in a repository, Mercurial will
remember the branch that your working directory on when you start a
command like \hgcmd{update} or \hgcmdargs{pull}{-u}.  It will update
the working directory to the tip of this branch, no matter what the
``repo-wide'' tip is.  To update to a revision that's on a different
named branch, you may need to use the \hgopt{update}{-C} option to
\hgcmd{update}.

This behaviour is a little subtle, so let's see it in action.  First,
let's remind ourselves what branch we're currently on, and what
branches are in our repository.
\interaction{branch-named.parents}
We're on the \texttt{bar} branch, but there also exists an older
\hgcmd{foo} branch.

We can \hgcmd{update} back and forth between the tips of the
\texttt{foo} and \texttt{bar} branches without needing to use the
\hgopt{update}{-C} option, because this only involves going backwards
and forwards linearly through our change history.
\interaction{branch-named.update-switchy}

If we go back to the \texttt{foo} branch and then run \hgcmd{update},
it will keep us on \texttt{foo}, not move us to the tip of
\texttt{bar}.
\interaction{branch-named.update-nothing}

Committing a new change on the \texttt{foo} branch introduces a new
head.
\interaction{branch-named.foo-commit}
We can no longer update from \texttt{foo} to \texttt{bar} without
going ``sideways'' in history, so Mercurial forces us to provide the
\hgopt{update}{-C} option to \hgcmd{update}.
\interaction{branch-named.update-bar}

\section{Branch names and merging}

As you've probably noticed, merges in Mercurial are not symmetrical.
Let's say our repository has two heads, 17 and 23.  If I
\hgcmd{update} to 17 and then \hgcmd{merge} with 23, Mercurial records
17 as the first parent of the merge, and 23 as the second.  Whereas if
I \hgcmd{update} to 23 and then \hgcmd{merge} with 17, it records 23
as the first parent, and 17 as the second.

This affects Mercurial's choice of branch name when you merge.  After
a merge, Mercurial will retain the branch name of the first parent
when you commit the result of the merge.  If your first parent's
branch name is \texttt{foo}, and you merge with \texttt{bar}, the
branch name will still be \texttt{foo} after you merge.

It's not unusual for a repository to contain multiple heads, each with
the same branch name.  Let's say I'm working on the \texttt{foo}
branch, and so are you.  We commit different changes; I pull your
changes; I now have two heads, each claiming to be on the \texttt{foo}
branch.  The result of a merge will be a single head on the
\texttt{foo} branch, as you might hope.

But if I'm working on the \texttt{bar} branch, and I merge work from
the \texttt{foo} branch, the result will remain on the \texttt{bar}
branch.
\interaction{branch-named.merge}

To give a more concrete example, if I'm working on the
\texttt{bleeding-edge} branch, and I want to bring in the latest fixes
from the \texttt{stable} branch, Mercurial will choose the ``right''
(\texttt{bleeding-edge}) branch name when I pull and merge from
\texttt{stable}.

\section{Branch naming is generally useful}

You shouldn't think of named branches as applicable only to situations
where you have multiple long-lived branches cohabiting in a single
repository.  They're very useful even in the one-branch-per-repository
case.  

In the simplest case, giving a name to each branch gives you a
permanent record of which branch a changeset originated on.  This
gives you more context when you're trying to follow the history of a
long-lived branchy project.

If you're working with shared repositories, you can set up a
\hook{pretxnchangegroup} hook on each that will block incoming changes
that have the ``wrong'' branch name.  This provides a simple, but
effective, defence against people accidentally pushing changes from a
``bleeding edge'' branch to a ``stable'' branch.  Such a hook might
look like this inside the shared repo's \hgrc.
\begin{codesample2}
  [hooks]
  pretxnchangegroup.branch = hg heads --template '{branches} ' | grep mybranch
\end{codesample2}

%%% Local Variables: 
%%% mode: latex
%%% TeX-master: "00book"
%%% End: 
