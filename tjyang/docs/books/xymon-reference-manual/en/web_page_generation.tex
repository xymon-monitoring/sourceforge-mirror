%%%%%%%%%%%%%%%%%%%%%%%%%%%%%%%%%%%%%%%%%%%%%%%%%%%%%%%%%%%%%%%%%%%%%%%%%%%%%%
%
%%%%%%%%%%%%%%%%%%%%%%%%%%%%%%%%%%%%%%%%%%%%%%%%%%%%%%%%%%%%%%%%%%%%%%%%%%%%%%
\chapter{Web page generation}

%%%%%%%%%%%%%%%%%%%%%%%%%%%%%%%%%%%%%%%%%%%%%%%%%%%%%%%%%%%%%%%%%%%%%%%%%%%%%%
%
%%%%%%%%%%%%%%%%%%%%%%%%%%%%%%%%%%%%%%%%%%%%%%%%%%%%%%%%%%%%%%%%%%%%%%%%%%%%%%
\section{BBGEN}

 bbgen - Hobbit webpage generator \subsection{SYNOPSIS}
\textbf{bbgen -?}
 
\textbf{bbgen --help}
 
\textbf{bbgen --version}
 
\textbf{bbgen [options] [output-directory]}
 
 (See the OPTIONS section for a description of the available commandline options). 

 
\subsection{DESCRIPTION}
\textbf{bbgen}
 generates the overview webpages for the Hobbit monitor. These are the
 webpages that show the overall status of your hosts, not the detailed
 status pages for each test. 


 
\subsection{OPTIONS}
 bbgen has a large number of commandline options. The options can be
 used to change the behaviour of bbgen and affect the web pages
 generated by it. 


 
\subsection{GENERAL OPTIONS}


 \begin{description}
\item[--help or -?] Provide a summary of available commandline options. 

 

\item[--version] Prints the version number of bbgen 

 

\item[--docurl=URL] Make hostnames be hyperlinks to documentation,
  accessed via a common web page (typically a CGI script or a
  PHP-driven dynamic page). The URL parameter is a formatting string
  with the name of the web page - you can put a ``\%s'' in it which
  will be replaced by the hostname being accessed. E.g. if you use the
  bb-notes extension from www.deadcat.net, you would enable this with
  ``--docurl=/hobbit/admin/notes.php?host=\%s''. For the host
  www.storner.dk this will result in a link to
  ``/hobbit/admin/notes.php?host=www.storner.dk``. 


 

\item[--doccgi=URL] This option is deprecated; please use --docurl instead. 

 

\item[--no-doc-window] By default, links to documentation for hosts
  and services cause a new window to appear with the information. With
  this option, the documentation will appear in the same window as the
  Hobbit status. 


 

\item[--htmlextension=.EXTENSION] Sets the filename extension used for
  the webpages generated by bbgen. By default, an extension of
  ``.html'' is used. Note that you need to specify the ``dot''. 


 

\item[--report[=COLUMNNAME]] With this option, bbgen will send a
  status message with details of how many hosts were processed, how
  many pages were generated, any errors that occurred during the run,
  and some timing statistics. The default columnname is ``bbgen''. 


 

\item[--htaccess[=htaccess-filename]] Create .htaccess files when new
  web page directories are created. The content of the .htaccess files
  are determined by the BBHTACCESS environment variable (for the
  top-level directory with bb.html and bb2.html); by the
  BBPAGEHTACCESS variable (for the page-level directories); and by the
  BBSUBPAGEHTACCESS variable for subpage- and subparent-level
  directories. The filename of the .htaccess files default to
  ``.htaccess'' if no filename is given with this option. The
  BBHTACCESS variable is copied verbatim into the top-level .htaccess
  file. The BBPAGEHTACCESS variable may contain a ``\%s'' where the
  name of the page is inserted. The BBSUBPAGEHTACCESS variable may
  contain two ``\%s'' instances: The first is replaced with the
  pagename, the second with the subpagename. 


 

\item[--max-eventcount=N] Limit the eventlog on the BB2 page to only N
  events. Default: 100. 


 

\item[--max-eventtime=N] Limit the eventlog on the BB2 page to events
  that happened within the past N minutes. Default: 240. 


 

\item[--no-eventlog] Disable the eventlog normally displayed on the BB2 page 

 

\item[--max-ackcount=N] Limit the acknowledgment log on the BB2 page to only N events. Default: 25. 

 

\item[--max-acktime=N] Limit the acknowledgment log on the BB2 page to acks that happened within the past N minutes. Default: 240. 

 

\item[--no-acklog] Disable the acknowledgement log normally displayed on the BB2 page. 

 

\item[--nklog[=NK log column]] This generates a text-based log of what
  is shown on the bbnk.html status page, and sends a status message
  for the BBDISPLAY server itself reflecting the color of the NK
  status page. This allows you to track when problems have appeared on
  the bbnk status page. The logfile is stored in \$BBHOME/nkstatus.log 



 


\end{description}

\subsection{PAGE LAYOUT OPTIONS}
 These options affect how the webpages generated by bbgen appear in the browser. 

 \begin{description}
\item[--pages-last] Put page- and subpage-links after hosts. 
\item[--pages-first] Put page- and subpage-links before hosts (default). 

  These two options decide whether a page with links to subpages and hosts have the hosts or the subpages first. 


 

\item[--subpagecolumns=N] Determines the number of columns used for links to pages and subpages. The default is N=1. 

 

\item[--maxrows=N] Column headings on a page are by default only shown
  at the beginning of a page, subpage or group of hosts. This options
  causes the column headings to repeat for every N hosts shown. 


 

\item[--pagetitle-links] Normally, only the colored ``dots'' next to a
  page or subpage act as links to the page itself. With this option,
  the page title will link to the page also. 


 

\item[--pagetext-headings] Use the description text from the ``page''
  or ``subpage'' tags as a heading for the page, instead of the
  ``Pages hosted locally'' or other standard heading. 


 

\item[--no-underline-headings] Normally, page headings are underlined
  using an HTML ``horizontal ruler'' tag. This option disables the
  underlining of headings. 


 

\item[--recentgifs[=MINUTES]] Use images named COLOR-recent.gif for
  tests, where the test status has changed within the past 24
  hours. These GIF files need to be installed in the
  \$BBHOME/www/gifs/ directory. By default, the threshold is set to 24
  hours - if you want it differently, you can specify the time limit
  also. E.g. ``--recentgifs=3h'' will show the recent GIFs for only 3
  hours after a status change. 


 

\item[--sort-group-only-items] In a normal ``group-only'' directive,
  you can specify the order in which the tests are displayed, from
  left to right. If you prefer to have the tests listed in
  alphabetical order, use this option - the page will then generate
  ``group-only'' groups like it generates normal groups, and sort the
  tests alphabetically. 


 

\item[--dialupskin=URL] If you want to visually show that a test is a
  dialup-test, you can use an alternate set of icons for the
  green/red/yellow$>$/etc. images by specifying this option. The URL
  parameter specified here overrides the normal setting from the
  BBSKIN environment variable, but only for dialup tests. 


 

\item[--reverseskin=URL] Same as ``--dialupskin'', but for reverse tests (tests with '!' in front). 

 

\item[--tooltips=[always,never,main]] Determines which pages use
  tooltips to show the description of the host (from the COMMENT entry
  in the \emph{bb-hosts(5)}

 file). If set to \textbf{always}
, tooltips are used on all pages. If set to \textbf{never}
, tooltips are never used. If set to \textbf{main}
, tooltips are used on the main pages, but not on the BB2 (all non-green) or NK (critical systems) pages. 

 


\end{description}

\subsection{COLUMN SELECTION OPTIONS}
 These options affect which columns (tests) are included in the webpages generated by bbgen. 

 \begin{description}
\item[--ignorecolumns=test[,test]] The given columns will be
  completely ignored by bbgen when generating webpages. Can be used to
  generate reports where you eliminate some of the more noisy tests,
  like ``msgs''. 


 

\item[--nk-reds-only] Only red status columns will be included on the
  NK page. By default, the NK page will contain hosts with red, yellow
  and clear status. 


 

\item[--bb2colors=COLOR[,COLOR]] Defines which colors cause a test to
  appear on the ``All non-green'' status page (a.k.a. the BB2
  page). COLOR is red, yellow or purple. The default is to include all
  three. 


 

\item[--bb2-ignorecolumns=test[,test]] Same as the --ignorecolumns, but applies to hosts on the BB2 page only. 

 

\item[--bb2-ignorepurples] Deprecated, use ``--bb2colors'' instead. 

 

\item[--bb2-ignoredialups] Ignore all dialup hosts on the BB2 page, including the BB2 eventlog. 

 

\item[--includecolumns=test[,test]] Always include these columns on
  bb2 page Will include certain columns on the bb2.html page,
  regardless of its color. Normally, bb2.html drops a test-column, if
  all tests are green. This can be used e.g. to always have a link to
  the trends column (with the RRD graphs) from your bb2.html page. 


 

\item[--eventignore=test[,test]] Ignore these tests in the BB2 event log display. 

 


\end{description}

\subsection{STATUS PROPAGATION OPTIONS}
 These options suppress the normal propagation of a status upwards in
 the page hierarchy. Thus, you can have a test with status yellow or
 red, but still have the entire page green. It is useful for tests
 that need not cause an alarm, but where you still want to know the
 actual status. These options set global defaults for all hosts; you
 can use the NOPROPRED and NOPROPYELLOW tags in the \emph{bb-hosts(5)}
 file to apply similar limits on a per-host basis. 


\begin{description}

\item[--nopropyellow=test[,test] or --noprop=test[,test]] Disable
  upwards status propagation when YELLOW. The ``--noprop'' option is
  deprecated and should not be used. 
 

\item[--noproppurple=test[,test]] Disable upwards status propagation when PURPLE. 

 

\item[--nopropred=test[,test]] Disable upwards status propagation when RED or YELLOW. 

 

\item[--nopropack=test[,test]] Disable upwards status propagation when
  status has been acknowledged. If you want to disable all acked tests
  from being propageted, use ``--nopropack=*''. 


 


\end{description}

\subsection{PURPLE STATUS OPTIONS}
 Purple statuses occur when reporting of a test status stops. A test
 status is valid for a limited amount of time - normally 30 minutes -
 and after this time, the test becomes purple. 


 \begin{description}
\item[--purplelog=FILENAME] Generate a logfile of all purple status messages. 

 


\end{description}

\subsection{ALTERNATE PAGESET OPTIONS}


 \begin{description}
\item[--pageset=PAGESETNAME] Build webpages for an alternate pageset than the default. See the PAGESETS section below. 

 

\item[--template=TEMPLATE] Use an alternate template for header and
  footer files. Typically used together the the ``--pageset'' option;
  see the PAGESETS section below. 

\end{description}

\subsection{ALTERNATE OUTPUT FORMATS}


 \begin{description}

\item[--wml[=test1,test2,...]] This option causes bbgen to generate a
  set of WML ``card'' files that can be accessed by a WAP device (cell
  phone, PDA etc.) The generated files contain the hosts that have a
  RED or YELLOW status on tests specified. This option can define the
  default tests to include - the defaults can be overridden or amended
  using the ``WML:'' or ``NK:'' tags in the \emph{bb-hosts(5)}
  file. If no tests are specified, all tests will be included. 


 

\item[--nstab=FILENAME] Generate an HTML file suitable for a Netscape
  6/Mozilla sidebar entry. To actually enable your users to obtain
  such a sidebar entry, you need this Javascript code in a webpage
  (e.g. you can include it in the \$BBHOME/web/bb\_header file): 


%  $<$SCRIPT TYPE=''text/javascript''$>$  
% $<$!--  
% function addNetscapePanel() \{  
% 
%if((typeofwindow.sidebar==''object'')\&\&  
%(typeofwindow.sidebar.addPanel==''function''))  
% 
%window.sidebar.addPanel(``Hobbit'',  
% 
%''\url{http://your.server.com/nstab.html}``,''``);  
% 
%else  
% 
%alert(``SidebaronlyforMozillaorNetscape6+'');  
% \}  
% //--$>$  
% $<$/SCRIPT$>$ 
%

  and then you can include a ``Add this to sidebar'' link using this as a template: 


% $<$AHREF=''javascript:addNetscapePanel();''$>$AddtoSidebar$<$/A$>$ 


  or if you prefer to have the standard Netscape ``Add tab'' button, you would do it with 

%
%  
%$<$AHREF=''javascript:addNetscapePanel();''$>$  
% 
%$<$IMGSRC=''/gifs/add-button.gif''HEIGHT=45WIDTH=100  
% 
%ALT=''[AddSidebar]''STYLE=''border:0''$>$  
% 
%$<$/A$>$ 


  The ``add-button.gif'' is available from Netscape at
  \url{http://developer.netscape.com/docs/manuals/browser/sidebar/add-button.gif.}



  If FILENAME does not begin with a slash, the Netscape sidebar file is placed in the \$BBHOME/www/ directory. 


 

\item[--nslimit=COLOR] The minimum color to include in the Netscape
  Sidebar - default is ``red'', meaning only critical alerts are
  included. If you want to include warnings also, use
  ``--nslimit=yellow''. 


 

\item[--rss] Generate RSS/RDF content delivery stream of your Hobbit
  alerts. This output format can be dynamically embedded in other web
  pages, much like the live newsfeeds often seen on web sites. Two RSS
  files will be generated, one reflects the BB2 page, the other
  reflects the BBNK page. They will be in the ``bb2.rss'' and
  ``bbnk.rss'' files, respectively. In addition, an RSS file will be
  generated for each page and/or subpage listing the hosts present on
  that page or subpage.  

 The FILENAME parameter previously allowed on the --rss option is now obsolete.  
 For more information about RSS/RDF content feeds, please see \url{http://www.syndic8.com/.}

 

\item[--rssextension=.EXTENSION] Sets the filename extension used for
  the RSS files generated by bbgen. By default, an extension of
  ``.rss'' is used. Note that you need to specify the ``dot''. 


 

\item[--rssversion= {0.91|0.92|1.0|2.0}] The desired output format of
  the RSS/RDF feed. Version 0.91 appears to be the most commonly used
  format, and is the default if this option is omitted. 


 

\item[--rsslimit=COLOR] The minimum color to include in the RSS feed -
  default is ``red'', meaning only critical alerts are included. If
  you want to include warnings also, use ``--rsslimit=yellow''. 

\end{description}

\subsection{OPTIONS USED BY CGI FRONT-ENDS}
\begin{description}
\item[--reportopts=START:END:DYNAMIC:STYLE] Invoke bbgen in
  report-generation mode. This is normally used by the
  \emph{bb-rep.cgi(1)}

 CGI script, but may also be used directly when pre-generating
 reports. The START parameter is the start-time for the report in Unix
 time\_t format (seconds since Jan 1st 1970 00:00 UTC); END is the
 end-time for the report; DYNAMIC is 0 for a pre-built report and 1
 for a dynamic (on-line) report; STYLE is ``crit'' to include only
 critical (red) events, ``non-crit'' to include all non-green events,
 and ``all'' to include all events. 


 

\item[--csv=FILENAME] Used together with --reportopts, this causes
  bbgen to generate an availability report in the form of a
  comma-separated values (CSV) file. This format is commonly used for
  importing into spreadsheets for further processing.  

 The CSV file includes Unix timestamps. To display these as human
 readable times in Excel, the formula
 \textbf{=C2/86400+DATEVALUE(1-jan-1970)}

 (if you have the Unix timestamp in the cell C2) can be used. The
 result cell should be formatted as a date/time field. Note that the
 timestamps are in UTC, so you may also need to handle local timezone
 and DST issues yourself. 


 

\item[--csvdelim=DELIMITER] By default, a comma is used to delimit
  fields in the CSV output. Some non-english spreadsheets use a
  different delimiter, typically semi-colon. To generate a CSV file
  with the proper delimiter, you can use this option to set the
  character used as delimiter. E.g. ``--csvdelim=;'' - note that this
  normally should be in double quotes, to prevent the Unix shell from
  interpreting the delimiter character as a commandline delimiter. 



\item[--snapshot=TIME] Generate a snapshot of the Hobbit pages, as
  they appeared at TIME. TIME is given as seconds since Jan 1st 1970
  00:00 UTC. Normally used via the \emph{bb-snapshot.cgi(1)} CGI script. 

 


\end{description}
\subsection{DEBUGGING OPTIONS}


 \begin{description}
\item[--debug] Causes bbgen to dump large amounts of debugging output
  to stdout, if it was compiled with the -DDEBUG enabled. When
  reporting a problem with bbgen, please try to reproduce the problem
  and provide the output from running bbgen with this option. 


 

\item[--timing] Dump information about the time spent by various parts
  of bbgen to stdout. This is useful to see what part of the
  processing is responsible for the run-time of bbgen.  

 Note: This information is also provided in the output sent to the Hobbit display when using the ``--report'' option. 

 


 


\end{description}

\subsection{BUILDING ALTERNATE PAGESETS}
 With version 1.4 of bbgen comes the possibility to generate multiple
 sets of pages from the same data.  

 Suppose you have two groups of people looking at the BB
 webpages. Group A wants to have the hosts grouped by the client, they
 belong to. This is how you have Hobbit set up - the default
 pageset. Now group B wants to have the hosts grouped by operating
 system - let us call it the ``os'' set. Then you would add the page
 layout to bb-hosts like this: 


 ospage win Microsoft Windows  
 ossubpage win-nt4 MS Windows NT 4  
 osgroup NT4 File servers  
 osgroup NT4 Mail servers  
 ossubpage win-xp MS Windows XP  
 ospage unix Unix  
 ossubpage unix-sun Solaris  
 ossubpage unix-linux Linux 


  This defines a set of pages with one top-level page (the bb.html
  page), two pages linked from bb.html (win.html and unix.html), and
  from e.g. the win.html page there are subpages win-nt4.html and
  win-xp.html  

 The syntax is identical to the normal ``page'' and ``subpage''
 directives in bb-hosts, but the directive is prefixed with the
 pageset name. Dont put any hosts in-between the page and subpage
 directives - just add all the directives at the top of the bb-hosts
 file.  

 How do you add hosts to the pages, then ? Simple - just put a tag
 ``OS:win-xp'' on the host definition line. The ``OS'' must be the
 same as prefix used for the pageset names, but in uppercase. The
 ``win-xp'' must match one of the pages or subpages defined within
 this pageset. E.g. 



  207.46.249.190 www.microsoft.com \# OS:win-xp \url{http://www.microsoft.com/} 
 64.124.140.181 www.sun.com \# OS:unix-sun \url{http://www.sun.com/}


  If you want the host to appear inside a group defined on that page,
  you must identify the group by number, starting at 1. E.g. to put a
  host inside the ``NT4 Mail servers'' group in the example above, use
  ``OS:win-nt4,2'' (the second group on the ``win-nt4'' page).  

 If you want the host to show up on the frontpage instead of a subpage, use ``OS:*'' . 


  All of this just defines the layout of the new pageset. To generate
  it, you must run bbgen once for each pageset you define -
  i.e. create an extension script like this: \begin{description}

\item[]\begin{verbatim}

#!/bin/sh

BBWEB="/hobbit/os" $BBHOME/bin/bbgen \
        --pageset=os --template=os \
        $BBHOME/www/os/

\end{verbatim}


\end{description}



  Save this to \$BBHOME/ext/os-display.sh, and set this up to run as a
  Hobbit extension; this means addng an extra section to
  hobbitlaunch.cfg to run it. 



  This generates the pages. There are some important options used here:  
 * BBWEB=''/hobbit/os'' environment variable, and the  
''\$BBHOME/www/os/''optionworktogether,andplacesthe  
newpagesetHTMLfilesinasubdirectoryoffthenormal  
Hobbitwebroot.IfyounormallyaccesstheHobbitpagesas  
''\url{http://hobbit.acme.com/hobbit/}``,youwillthenaccess  
thenewpagesetas''\url{http://hobbit.acme.com/hobbit/os/}``  
NB:ThedirectorygivenasBBWEBmustcontainasymbolic  
linktothe\$BBHOME/www/html/directory,orlinksto  
individualstatusmessageswillnotwork.Similarlinks  
shouldbemadeforthegifs/,help/andnotes/  
directories.  
 * ``--pageset=os'' tells bbgen to structure the webpages  
usingthe''os''layout,insteadofthedefaultlayout.  
 * ``--template=os'' tells bbgen to use a different set of  
header-andfooter-templates.Normallybbgenusesthe  
standardtemplatein\$BBHOME/web/bb\_headerand  
.../bb\_footer-withthisoption,itwillinsteaduse  
thefiles''os\_header''and''os\_footer''fromthe  
\$BBHOME/web/directory.Thisallowsyoutocustomize  
headersandfootersforeachpageset.Ifyoujustwant  
tousethenormaltemplate,youcanomitthisoption. 


 
\subsection{USING BBGEN FOR REPORTS}
 bbgen reporting is implemented via drop-in replacements for the standard Hobbit reporting scripts (bb-rep.sh and bb-replog.sh) installed in your webservers cgi-bin directory. 

  These two shell script have been replaced with two very small
  shell-scripts, that merely setup the Hobbit environment variables,
  and invoke the \emph{bb-rep.cgi(1)} or \emph{bb-replog.cgi(1)}
  scripts in \$BBHOME/bin/ 



  You can use bbgen commandline options when generating reports,
  e.g. to exclude certain types of tests
  (e.g. ``--ignorecolumns=msgs'') from the reports, to specify the
  name of the trends- and info- columns that should not be in the
  report, or to format the report differently
  (e.g. ``--subpagecolumns=2''). If you want certain options to be
  used when a report is generated from the web interface, put these
  options into your \$BBHOME/etc/hobbitserver.cfg file in the
  BBGENREPOPTS environment variable. 



  The report files generated by bbgen are stored in individual
  directories (one per report) below the \$BBHOME/www/rep/
  directory. These should be automatically cleaned up - as new reports
  are generated, the old ones get removed. 



  After installing, try generating a report. You will probably see
  that the links in the upper left corner (to bb-ack.html, bb2.html
  etc.) no longer works. To fix these, change your
  \$BBHOME/web/bbrep\_header file so these links do not refer to
  ``\&BBWEB'' but to the normal URL prefix for your Hobbit pages. 



 
\subsection{SLA REPORTING}
 bbgen reporting allows for the generation of true \index{SLA} ( \index{Service Level
 Agreement}) reports, also for service periods that are not 24x7. This
 is enabled by defining a ``REPORTTIME:timespec'' tag for the hosts to
 define the service period, and optionally a ``WARNPCT:level'' tag to
 define the agreed availability. 


  Note: See \emph{bb-hosts(5)}
 for the exact syntax of these options. 


  ``REPORTTIME:timespec'' specifies the time of day when the service
  is expected to be up and running. By default this is 24 hours a day,
  all days of the week. If your SLA only covers Mon-Fri 7am - 8pm, you
  define this as ``REPORTTIME=W:0700:2000'', and the report generator
  will then compute both the normal 24x7 availability but also a ``SLA
  availability'' which only takes the status of the host during the
  SLA period into account. 



  The DOWNTIME:timespec parameter affects the SLA availability
  calculation. If an outage occurs during the time defined as possible
  ``DOWNTIME'', then the failure is reported with a status of
  ``blue''. (The same color is used if you ``disable'' then host using
  the Hobbit ``disable'' function). The time when the test status is
  ``blue'' is not included in the SLA calculation, neither in the
  amount of time where the host is considered down, nor in the total
  amount of time that the report covers. So ``blue'' time is
  effectively ignored by the SLA availability calculation, allowing
  you to have planned downtime without affecting the reported SLA
  availability. 



  Example: A host has ``DOWNTIME:*:0700:0730 REPORTTIME=W:0600:2200''
  because it is rebooted every day between 7am and 7.30am, but the
  service must be available from 6am to 10pm. For the day of the
  report, it was down from 7:10am to 7:15am (the planned reboot), but
  also from 9:53pm to 10:15pm. So the events for the day are: 



  
0700:greenfor10minutes(600seconds)  
0710:bluefor5minutes(300seconds)  
0715:greenfor14hours38minutes(52680seconds)  
2153:redfor22minutes(1320seconds)  
2215:green 


  The service is available for 600+52680 = 53280 seconds. It is down
  (red) for 420 seconds (the time from 21:53 until 22:00 when the SLA
  period ends). The total time included in the report is 15 hours (7am
  - 10pm) except the 5 minutes blue = 53700 seconds. So the SLA
  availability is 53280/53700 = 99,22\% 



  The ``WARNPCT:level'' tag is supported in the bb-hosts file, to set
  the availability threshold on a host-by-host basis. This threshold
  determines whether a test is reported as green, yellow or red in the
  reports. A default value can be set for all hosts with the via the
  BBREPWARN environment variable, but overridden by this tag. The
  level is given as a percentage, e.g. ``WARNPCT:98.5'' 

 
\subsection{PRE-GENERATED REPORTS}
 Normally, bbgen produce reports that link to dynamically generated webpages with the detailed status of a test (via the bb-replog.sh CGI script). 

  It is possible to have bbgen produce a report without these dynamic
  links, so the report can be exported to another server. It may also
  be useful to pre-generate the reports, to lower the load by having
  multiple users generate the same reports. 



  To do this, you must run bbgen with the ``--reportopts'' option to
  select the time interval that the report covers, the reporting style
  (critical, non-green, or all events), and to request that no dynamic
  pages are to be generated. 



  The syntax is: 


  
bbgen--reportopts=starttime:endtime:nodynamic:style 


  ``starttime'' and ``endtime'' are specified as Unix time\_t values,
  i.e. seconds since Jan 1st 1970 00:00 GMT. Fortunately, this can
  easily be computed with the GNU date utility if you use the ``+\%s''
  output option. If you don't have the GNU date utility, either pick
  that up from www.gnu.org; or you can use the ``etime'' utility for
  the same purpose, which is available from the archive at
  www.deadcat.net. 



  ``nodynamic'' is either 0 (for dynamic pages, the default) or 1 (for no dynamic, i.e. pre-generated, pages). 


  ``style'' is either ``crit'' (include critical i.e. red events only), ``nongr'' (include all non-green events), or ``all'' (include all events). 


  Other bbgen options can be used, e.g. ``--ignorecolumns'' if you want to exclude certain tests from the report. 


  You will normally also need to specify the BBWEB environment
  variable (it must match the base URL for where the report will be
  made accessible from), and an output directory where the report
  files are saved. If you specify BBWEB, you should probably also
  define the BBHELPSKIN and BBNOTESSKIN environment variables. These
  should point to the URL where your Hobbit help- and notes-files are
  located; if they are not defined, the links to help- and notes-files
  will point inside the report directory and will probably not work. 



  So a typical invocation of bbgen for a static report would be: 


  
START=`date+\%s--date=''22Jun200300:00:00''`  
END=`date+\%s--date=''22Jun200323:59:59''`  
BBWEB=/reports/bigbrother/daily/2003/06/22$\backslash$  
BBHELPSKIN=/hobbit/help$\backslash$  
BBNOTESSKIN=/hobbit/notes$\backslash$  
bbgen--reportopts=\$START:\$END:1:crit$\backslash$  
--subpagecolumns=2$\backslash$  
/var/www/docroot/reports/hobbit/daily/2003/06/22 


  The ``BBWEB'' setting means that the report will be available with a
  URL of
  ``\url{http://www.server.com/reports/hobbit/daily/2003/06/22}``. The
  report contains internal links that use this URL, so it cannot be
  easily moved to another location. 



  The last parameter is the corresponding physical directory on your
  webserver matching the BBWEB URL. You can of course create the
  report files anywhere you like - perhaps on another machine - and
  then move them to the webserver later on. 



  Note how the \emph{date(1)}
 utility is used to calculate the start- and end-time parameters. 


 
\subsection{SEE ALSO}
bb-hosts(5), hobbitserver.cfg(5), hobbitlaunch.cfg(5), bb-rep.cgi(1), bb-snapshot.cgi(1), hobbit(7) 


%%%%%%%%%%%%%%%%%%%%%%%%%%%%%%%%%%%%%%%%%%%%%%%%%%%%%%%%%%%%%%%%%%%%%%%%%%%%%%
%
%%%%%%%%%%%%%%%%%%%%%%%%%%%%%%%%%%%%%%%%%%%%%%%%%%%%%%%%%%%%%%%%%%%%%%%%%%%%%%
\newpage
\section{BB-HIST.CGI}

 bb-hist.cgi - CGI program to display service history \subsection{SYNOPSIS}
\textbf{bb-hist.cgi}


 
\subsection{DESCRIPTION}
\textbf{bb-hist.cgi}
 is invoked as a CGI script via the bb-hist.sh CGI wrapper. It is
 passed a QUERY\_STRING environment variable with the following
 parameters: 


  
HISTFILE(aHobbitservicehistoryfile)  
ENTRIES(thenumberofentriestoshow)  
 The following non-standard parameters are handled by the bbgen version of bb-hist.cgi: 


  
IP(IPaddressofhost-fordisplaypurposesonly)  
PIXELS(widthofcolorbarwheninpixel-mode)  
ENDTIME(whenthecolorbarbegins,atime\_tvalue)  
BARSUMS(whichcolorbarsandsummariestoshow) 


  bb-hist.cgi analyses the service history file for changes that have
  occurred within the past 24 hours, and build a colorbar showing the
  status of the service over this period of time. A statistics summary
  is also produced, listing the amount of time for each status (green,
  yellow, red, purple, blue, clear). 



  Finally, a summary of the last N events is given, with links to the actual event logs. 


  Unlike the standard bb-hist.sh script, bb-hist.cgi provides a
  colorbar and statistics summaries also covering the past 1 week, 4
  weeks and 1 year of data. Via links it is possible to browse the
  entire history of the service at the requested interval. 



  Note that since the resolution of the display is limited, events may
  be too short to show up on a colorbar; also, the exact placement of
  an event may not fully match up with the time-markers. 



  The graphs should correctly handle the display of months with
  different number of days, as well as the display of periods that
  involve beginning and end of Daylight Savings Time, if this occurs
  in your timezone. 



  All dates and times shown are in local time for the timezone defined on the BBDISPLAY server. 


 


 
\subsection{PARAMETERS}
\begin{description}

\item[HISTFILE] Defines the host and service whose history is
  presented. 

\item[ENTRIES] The number of log-entries to show in the event log
  table. Default is 50; to view all log entries set this to ``ALL''. 

\item[IP] The IP-address of the host. This is only used for the title
  of the document. 

\item[PIXELS] The width of the colorbar graph in pixels. If this is
  set to 0, a percentage-based graph will be shown, similar to the one
  provided by the standard bb-hist.sh script. Pixel-based graphs can
  have a higher resolution, but do not resize automatically to suit
  the size of a browser window. The default value for this parameter
  is defined at compile-time; 960 is a good value for displays with a
  1024x768 resolution. 

\item[BARSUMS] Defines which colorbars and summaries to show. This is
  a number made up from a bitmask. The 1-day graph uses the value
  ``1''; the 1-week graph uses the value ``2''; the 4-week graph uses
  the value ``4'' and the 1-year graph the value ``8''. To show
  multiple graph, add the values - e.g. ``6'' will show the 1-week and
  4-weeks graphs, whereas ``15'' will show all the graphs. The default
  is defined at compile-time. 

\item[ENDTIME] The history display by default ends with the current
  time. Setting the ENDTIME parameter causes it to end at the time
  specified - this is given as a Unix ``time\_t'' value, i.e. as the
  number of seconds elapsed since Jan 1 1970 00:00 UTC. 


 


\end{description}
\subsection{OPTIONS}
\begin{description}
\item[--env=FILENAME] Load the environment from FILENAME before executing the CGI. 

\end{description}
\subsection{SEE ALSO}
bb-hosts(5), hobbitserver.cfg(5) 

 

%%%%%%%%%%%%%%%%%%%%%%%%%%%%%%%%%%%%%%%%%%%%%%%%%%%%%%%%%%%%%%%%%%%%%%%%%%%%%%
%
%%%%%%%%%%%%%%%%%%%%%%%%%%%%%%%%%%%%%%%%%%%%%%%%%%%%%%%%%%%%%%%%%%%%%%%%%%%%%%
\newpage
\section{HOBBIT-NKVIEW.CGI}


 hobbit-nkview.cgi - Hobbit Critical Systems view CGI 

\subsection{SYNOPSIS}
\textbf{hobbit-nkview.cgi}


 
\subsection{DESCRIPTION}
\textbf{hobbit-nkview.cgi} is invoked as a CGI script via the hobbit-nkview.sh CGI wrapper. 

  hobbit-nkview.cgi matches the current critical statuses against the
  \emph{hobbit-nkview.cfg(5)} file, and generates the ``Critical
  Systems'' view. 



 
\subsection{RELATION TO BBNK PAGE}
 This view is a replacement for the statically generated ``bbnk'' page
 provided in versions of Hobbit prior to version 4.2. Although the
 ``bbnk'' pages are supported throughout Hobbit 4.x, it is recommended
 that You switch to the newer Critical Systems view provided by this
 CGI. 


 
\subsection{OPTIONS}
\begin{description}
\item[--nkacklevel=NUMBER] Sets the acknowledgment level for
  acknowledgments sent via the \emph{hobbit-ackinfo.cgi(1)} page. Note
  that this may be overridden by the configuration of the
  hobbit-ackinfo.cgi utility. 


 

\item[--env=FILENAME] Loads the environment defined in FILENAME before
  executing the CGI script. 


 

\item[--area=NAME] Load environment variables for a specific area. NB:
  if used, this option must appear before any --env=FILENAME option. 


 

\item[--debug] Enables debugging output. 

 


\end{description}
\subsection{ENVIRONMENT VARIABLES}
\begin{description}
\item[BBHOME] Used to locate the template files for the generated web pages. 

 

\item[QUERY\_STRING] Contains the parameters for the CGI script. 


\end{description}
\subsection{SEE ALSO}
hobbit-ackinfo.cgi(1), hobbit(7) 


%%%%%%%%%%%%%%%%%%%%%%%%%%%%%%%%%%%%%%%%%%%%%%%%%%%%%%%%%%%%%%%%%%%%%%%%%%%%%%
%
%%%%%%%%%%%%%%%%%%%%%%%%%%%%%%%%%%%%%%%%%%%%%%%%%%%%%%%%%%%%%%%%%%%%%%%%%%%%%%
\newpage
\section{BBHOSTGREP}

 bbhostgrep - pick out lines in bb-hosts

 \subsection{SYNOPSIS}
\textbf{bbhostgrep --help}
 
\textbf{bbhostgrep --version}
 
\textbf{bbhostgrep [--noextras] [--test-untagged] [--bbdisp] [--bbnet] TAG [TAG...]}


 
\subsection{DESCRIPTION}
\emph{bbhostgrep(1)}
 is for use by extension scripts that need to pick out the entries in
 a bb-hosts file that are relevant to the script. 


  The utility accepts test names as parameters, and will then parse
  the bb-hosts file and print out the host entries that have at least
  one of the wanted tests specified. Tags may be given with a trailing
  asterisk '*', e.g. ``bbhostgrep http*'' is needed to find all http
  and https tags. 



  The bbhostgrep utility supports the use of ``include'' directives
  inside the bb-hosts file, and will find matching tags in all
  included files. 



  If the DOWNTIME or SLA tags are used in the \emph{bb-hosts(5)} file,
  these are interpreted relative to the current time. bbhostgrep then
  outputs a ``INSIDESLA'' or ``OUTSIDESLA'' tag for easier use by
  scripts that want to check if the current time is inside or outside
  the expected uptime window. 



 
\subsection{OPTIONS}
\begin{description}
\item[--noextras] Remove the ``testip'', ``dialup'', ``INSIDESLA'' and ``OUTSIDESLA'' tags from the output. 

 

\item[--test-untagged] When using the BBLOCATION environment variable
  to test only hosts on a particular network segment, bbtest-net will
  ignore hosts that do not have any ``NET:x'' tag. So only hosts that
  have a NET:\$BBLOCATION tag will be tested.   With this option,
  hosts with no NET: tag are included in the test, so that all hosts
  that either have a matching NET: tag, or no NET: tag at all are
  tested. 


 

\item[--no-down[=TESTNAME]] bbhostgrep will query the Hobbit server
  for the current status of the ``conn'' test, and if TESTNAME is
  specified also for the current state of the specified test. If the
  status of the ``conn'' test for a host is non-green, or the status
  of the TESTNAME test is disabled, then this host is ignored and will
  not be included in the output. This can be used to ignore hosts that
  are down, or hosts where the custom test is disabled. 


 

\item[--bbdisp] Search the bb-hosts file following include statements
  as a BBDISPLAY server would. 


 

\item[--bbnet] Search the bb-hosts file following include statements
  as a BBNET server would. 


 


\end{description}
\subsection{EXAMPLE}
 If your bb-hosts file looks like this 

  
192.168.1.1www.test.com\#ftptelnet!oracle  
192.168.1.2db1.test.com\#oracle  
192.168.1.3mail.test.com\#smtp 


  and you have a custom Hobbit extension script that performs the
  ``oracle'' test, then running ``bbhostgrep oracle'' would yield 



  
192.168.1.1www.test.com\#!oracle  
192.168.1.2db1.test.com\#oracle 


  so the script can quickly find the hosts that are of interest. 


  Note that the reverse-test modifier - ``!oracle'' - is included in
  the output; this also applies to the other test modifiers defined by
  Hobbit (the dialup and always-true modifiers). 



  If your extension scripts use more than one tag, just list all of
  the interesting tags on the command line. 



  bbhostgrep also supports the ``NET:location'' tag used by
  bbtest-net, so if your script performs network checks then it will
  see only the hosts that are relevant for the test location that the
  script currently executes on. 



 
\subsection{USE IN EXTENSION SCRIPTS}
 To integrate bbhostgrep into an existing script, look for the line in the script that grep's in the \$BBHOSTS file. Typically it will look somewhat like this: 

  
\$GREP-i''\^{}[0-9].*\#.*TESTNAME''\$BBHOSTS|...codetohandletest 


  Instead of the grep, we will use bbhostgrep. It then becomes 


  
\$BBHOME/bin/bbhostgrepTESTNAME|...codetohandletest 


  which is simpler, less error-prone and more efficient. 


 
\subsection{ENVIRONMENT VARIABLES}
\begin{description}
\item[BBLOCATION] If set, bbhostgrep outputs only lines from bb-hosts that have a matching NET:\$BBLOCATION setting. 

 

\item[BBHOSTS] Filename for the Hobbit \emph{bb-hosts(5)}
 file. 

 


\end{description}
\subsection{FILES}
\begin{description}
\item[\$BBHOSTS] The Hobbit bb-hosts file 

 


\end{description}
\subsection{SEE ALSO}
bb-hosts(5), hobbitserver.cfg(5) 

  
%%%%%%%%%%%%%%%%%%%%%%%%%%%%%%%%%%%%%%%%%%%%%%%%%%%%%%%%%%%%%%%%%%%%%%%%%%%%%%
%
%%%%%%%%%%%%%%%%%%%%%%%%%%%%%%%%%%%%%%%%%%%%%%%%%%%%%%%%%%%%%%%%%%%%%%%%%%%%%%
\newpage
\section{HOBBITD}

 hobbitd - Master network daemon for a Hobbit server
 \subsection{SYNOPSIS}
\textbf{hobbitd [options]}


 
\subsection{DESCRIPTION}
 hobbitd is the core daemon in the Hobbit Monitor. It is designed to handle monitoring of a large number of hosts, with a strong focus on being a high-speed, low-overhead implementation of a Big Brother compatible server. 

  To achieve this, hobbitd stores all information about the state of the monitored systems in memory, instead of storing it in the host filesystem. A number of plug-ins can be enabled to enhance the basic operation; e.g. a set of plugins are provided to implement persistent storage in a way that is compatible with the Big Brother daemon. However, even with these plugins enabled, hobbitd still performs much faster than the standard bbd daemon. 


  hobbitd is normally started and controlled by the \emph{hobbitlaunch(8)}
 tool, and the command used to invoke hobbitd should therefore be in the hobbitlaunch.cfg file. 


 
\subsection{OPTIONS}
\begin{description}
\item[--bbhosts=FILENAME] Specifies the path to the Hobbit bb-hosts file. This is used to check if incoming status messages refer to known hosts; depending on the ``--ghosts'' option, messages for unknown hosts may be dropped. If this option is omitted, the default path used is set by the BBHOSTS environment variable. 

 

\item[--checkpoint-file=FILENAME] With regular intervals, hobbitd will dump all of its internal state to this check-point file. It is also dumped when hobbitd terminates, or when it receives a SIGUSR1 signal. 

 

\item[--checkpoint-interval=N] Specifies the interval (in seconds) between dumps to the check-point file. The default is 900 seconds (15 minutes). 

 

\item[--restart=FILENAME] Specifies an existing file containing a previously generated hobbitd checkpoint. When starting up, hobbitd will restore its internal state from the information in this file. You can use the same filename for ``--checkpoint-file'' and ``--restart''. 

 

\item[--ghosts= {allow|drop|log}] How to handle status messages from unknown hosts. The ``allow'' setting accepts all status messages, regardless of whether the host is known in the bb-hosts file or not. ``drop'' silently ignores reports from unknown hosts. ``log'' works like drop, but logs the event in the hobbitd output file. The default is ``allow'', unless the BBGHOSTS environment variable is set. 

 

\item[--no-purple] Prevent status messages from going purple when they are no longer valid. Unlike the standard bbd daemon, purple-handling is done by hobbitd. 

 

\item[--listen=IP[:PORT]] Specifies the IP-address and port where hobbitd will listen for incoming connections. By default, hobbitd listens on IP 0.0.0.0 (i.e. all IP- adresses available on the host) and port 1984. 

 

\item[--daemon] hobbitd is normally started by \emph{hobbitlaunch(8)}
 it will then detach from the terminal and continue running as a background task. 

 

\item[--timeout=N] Set the timeout used for incoming connections. If a status has not been received more than N seconds after the connection was accepted, then the connection is dropped and any status message is discarded. Default: 10 seconds. 

 

\item[--env=FILENAME] Loads the content of FILENAME as environment settings before starting hobbitd. This is mostly used when running as a stand-alone daemon; if hobbitd is started by hobbitlaunch, the environment settings are controlled by the hobbitlaunch hobbitlaunch.cfg file. 

 

\item[--pidfile=FILENAME] hobbitd writes the process-ID it is running with to this file. This is for use in automated startup scripts. The default file is \$BBSERVERLOGS/hobbitd.pid. 

 

\item[--log=FILENAME] Redirect all output from hobbitd to FILENAME. 

 

\item[--store-clientlogs[=[!]COLUMN]] Determines which status columns can cause a client message to be broadcast to the CLICHG channel. By default, no client messages are pushed to the CLICHG channel. If this option is specified with no parameter list, all status columns that go into an alert state will trigger the client data to be sent to the CLICHG channel. If a paramater list is added to this option, only those status columns listed in the list will cause the client data to be sent to the CLICHG channel. Several column names can be listed, separated by commas. If all columns are given as ``!COLUMNNAME'', then all status columns except those listed will cause the client data to be sent. 

 

\item[--status-senders=IP[/MASK][,IP/MASK]] Controls which hosts may send ``status'', ``combo'', ``config'' and ``query'' commands to hobbitd. 

  By default, any host can send status-updates. If this option is used, then status-updates are accepted only if they are sent by one of the IP-adresses listed here, or if they are sent from the IP-address of the host that the updates pertains to (this is to allow Hobbit clients to send in their own status updates, without having to list all clients here). So typically you will need to list your BBNET servers here. 


  The format of this option is a list of IP-adresses, optionally with a network mask in the form of the number of bits. E.g. if you want to accept status-updates from the host 172.16.10.2, you would use  
 
--status-senders=172.16.10.2  
 whereas if you want to accept status updates from both 172.16.10.2 and from all of the hosts on the 10.0.2.* network (a 24-bit IP network), you would use  
 
--status-senders=172.16.10.2,10.0.2.0/24 


 

\item[--maint-senders=IP[/MASK][,IP/MASK]] Controls which hosts may send maintenance commands to hobbitd. Maintenance commands are the ``enable'', ``disable'', ``ack'' and ``notes'' commands. Format of this option is as for the --status-senders option. It is strongly recommended that you use this to restrict access to these commands, so that monitoring of a host cannot be disabled by a rogue user - e.g. to hide a system compromise from the monitoring system. 

 \textbf{Note:}
 If messages are sent through a proxy, the IP-address restrictions are of little use, since the messages will appear to originate from the proxy server address. It is therefore strongly recommended that you do NOT include the address of a server running bbproxy in the list of allowed addresses. 


 

\item[--www-senders=IP[/MASK][,IP/MASK]] Controls which hosts may send commands to retrieve the state of hobbitd. These are the ``hobbitdlog'', ``hobbitdboard'' and ``hobbitdxboard'' commands, which are used by \emph{bbgen(1)}
 and \emph{bbcombotest(1)}
 to retrieve the state of the Hobbit system so they can generate the Hobbit webpages. 

 \textbf{Note:}
 If messages are sent through a proxy, the IP-address restrictions are of little use, since the messages will appear to originate from the proxy server address. It is therefore strongly recommended that you do NOT include the address of a server running bbproxy in the list of allowed addresses. 


 

\item[--admin-senders=IP[/MASK][,IP/MASK]] Controls which hosts may send administrative commands to hobbitd. These commands are the ``drop'' and ``rename'' commands. Access to these should be restricted, since they provide an un-authenticated means of completely disabling monitoring of a host, and can be used to remove all traces of e.g. a system compromise from the Hobbit monitor. 

 \textbf{Note:}
 If messages are sent through a proxy, the IP-address restrictions are of little use, since the messages will appear to originate from the proxy server address. It is therefore strongly recommended that you do NOT include the address of a server running bbproxy in the list of allowed addresses. 


 

\item[--no-download] Disable the ``download'' and ``config'' commands which can be used by clients to pull files from the Hobbit server. The use of these may be seen as a security risk since they allow file downloads. 

 

\item[--debug] Enable debugging output. 

 

\item[--dbghost=HOSTNAME] For troubleshooting problems with a specific host, it may be useful to track the exact communications from a single host. This option causes hobbitd to dump all traffic from a single host to the file ``/tmp/hobbitd.dbg''. 

 


\end{description}
\subsection{HOW ALERTS TRIGGER}
 When a status arrives, hobbitd matches the old and new color against the ``alert'' colors (from the ``ALERTCOLORS'' setting) and the ``OK'' colors (from the ``OKCOLORS'' setting). The old and new color falls into one of three categories: 

 \textbf{OK:}
 The color is one of the ``OK'' colors (e.g. ``green''). 


 \textbf{ALERT:}
 The color is one of the ``alert'' colors (e.g. ``red''). 


 \textbf{UNDECIDED:}
 The color is neither an ``alert'' color nor an ``OK'' color (e.g. ``yellow''). 


  If the new status shows an ALERT state, then a message to the \emph{hobbitd\_alert(8) }
 module is triggered. This may be a repeat of a previous alert, but \emph{hobbitd\_alert(8)}
 will handle that internally, and only send alert messages with the interval configured in \emph{hobbit-alerts.cfg(5).}



  If the status goes from a not-OK state (ALERT or UNDECIDED) to OK, and there is a record of having been in a ALERT state previously, then a recovery message is triggered. 


  The use of the OK, ALERT and UNDECIDED states make it possible to avoid being flooded with alerts when a status flip-flops between e.g yellow and red, or green and yellow. 


 
\subsection{CHANNELS}
 A lot of functionality in the Hobbit server is delegated to ``worker modules'' that are fed various events from hobbitd via a ``channel''. Programs access a channel using IPC mechanisms - specifically, shared memory and semaphores - or by using an instance of the \emph{hobbitd\_channel(8)}
 intermediate program. hobbitd\_channel enables access to a channel via a simple file I/O interface. 

  A skeleton program for hooking into a hobbitd channel is provided as part of Hobbit in the \emph{hobbitd\_sample(8)}
 program. 


  The following channels are provided by hobbitd: 


 \textbf{status}
 This channel is fed the contents of all incoming ``status'' and ``summary'' messages. 


 \textbf{stachg}
 This channel is fed information about tests that change status, i.e. the color of the status-log changes. 


 \textbf{page}
 This channel is fed information about tests where the color changes between an alert color and a non-alert color. It also receives information about ``ack'' messages. 


 \textbf{data}
 This channel is fed information about all ``data'' messages. 


 \textbf{notes}
 This channel is fed information about all ``notes'' messages. 


 \textbf{enadis}
 This channel is fed information about hosts or tests that are being disabled or enabled. 


 \textbf{client}
 This channel is fed the contents of the client messages sent by Hobbit clients installed on the monitored servers. 


 \textbf{clichg}
 This channel is fed the contents of a host client messages, whenever a status for that host goes red, yellow or purple. 


  Information about the data stream passed on these channels is in the Hobbit source-tree, see the ``hobbitd/new-daemon.txt'' file. 


 
\subsection{SIGNALS}
\begin{description}
\item[SIGHUP] Re-read the bb-hosts configuration file. 

 

\item[SIGUSR1] Force an immediate dump of the checkpoint file. 

 


\end{description}
\subsection{BUGS}
 Timeout of incoming connections are not strictly enforced. The check for a timeout only triggers during the normal network handling loop, so a connection that should timeout after N seconds may persist until some activity happens on another (unrelated) connection. 

 
\subsection{FILES}
 If ghost-handling is enabled via the ``--ghosts'' option, the bb-hosts file is read to determine the names of all known hosts. 

 
\subsection{SEE ALSO}
hobbit(7), hobbitserver.cfg(5). 
  

%%%%%%%%%%%%%%%%%%%%%%%%%%%%%%%%%%%%%%%%%%%%%%%%%%%%%%%%%%%%%%%%%%%%%%%%%%%%%%
%
%%%%%%%%%%%%%%%%%%%%%%%%%%%%%%%%%%%%%%%%%%%%%%%%%%%%%%%%%%%%%%%%%%%%%%%%%%%%%%
\newpage
\section{HOBBITPING}

 hobbitping - Hobbit ping tool
 \subsection{SYNOPSIS}
\textbf{hobbitping [--retries=N] [--timeout=N] [IP-adresses]}


 
\subsection{DESCRIPTION}
\emph{hobbitping(1)}
 is used for ping testing of the hosts monitored by the
 \emph{hobbit(7)} monitoring system. It reads a list of IP adresses
 from stdin, and performs a ``ping'' check to see if these hosts are
 alive. It is normally invoked by the \emph{bbtest-net(1)} utility,
 which performs all of the Hobbit network tests. 


  Optionally, if a list of IP-adresses is passed as commandline
  arguments, it will ping those IP's instead of reading them from
  stdin. 



  hobbitping only handles IP-adresses, not hostnames. 


  hobbitping was inspired by the \emph{fping(1)} tool, but has been
  written from scratch to implement a fast ping tester without much of
  the overhead found in other such utilities. The output from
  hobbitping is similar to that of ``fping -Ae''. 



  hobbitping probes multiple systems in parallel, and the runtime is
  therefore mostly dependant on the timeout-setting and the number of
  retries. With the default options, hobbitping takes approximately 18
  seconds to ping all hosts (tested with an input set of 1500 IP
  adresses). 



 
\subsection{SUID-ROOT INSTALLATION REQUIRED}
 hobbitping needs to be installed with suid-root privileges, since it
 requires a ``raw socket'' to send and receive ICMP Echo (ping)
 packets. 


  hobbitping is implemented such that it immediately drops the root
  privileges, and only regains them to perform two operations:
  Obtaining the raw socket, and optionally binding it to a specific
  source address. These operations are performed as root, the rest of
  the time hobbitping runs with normal user privileges. Specifically,
  no user-supplied data or network data is used while running with
  root privileges. Therefore it should be safe to provide hobbitping
  with the necessary suid-root privileges. 



 
\subsection{OPTIONS}
\begin{description}
\item[--retries=N] Sets the number of retries for hosts that fail to
  respond to the initial ping, i.e. the number of ping probes sent in
  addition to the initial probe. The default is --retries=2, to ping a
  host 3 times before concluding that it is not responding. 


 

\item[--timeout=N] Determines the timeout (in seconds) for ping
  probes. If a host does not respond within N seconds, it is regarded
  as unavailable, unless it responds to one of the retries. The
  default is --timeout=5. 


 

\item[--responses=N] hobbitping normally stops pinging a host after
  receiving a single response, and uses that to determine the
  round-trip time. If the first response takes longer to arrive -
  e.g. because of additional network overhead when first determining
  the route to the target host - it may skew the round-trip-time
  reports. You can then use this option to require N responses, and
  hobbitping will calculate the round-trip time as the average of all
  of responsetimes. 


 

\item[--max-pps=N] Maximum number of packets per second. This limits
  the number of ICMP packets hobbitping will send per second, by
  enforcing a brief delay after each packet is sent. The default
  setting is to send a maximum of 50 packets per second. Note that
  increasing this may cause flooding of the network, and since ICMP
  packets can be discarded by routers and other network equipment,
  this can cause erratic behaviour with hosts recorded as not
  responding when they are in fact OK. 


 

\item[--source=ADDRESS] Use ADDRESS as the source IP address of the
  ping packets sent. On multi-homed systems, allows you to select the
  source IP of the hosts going out, which might be necessary for ping
  to work. 


 

\item[--debug] Enable debug output. This prints out all packets sent and received. 



\end{description}
\subsection{SEE ALSO}
hobbit(7), bbtest-net(1), fping(1) 

 
%%%%%%%%%%%%%%%%%%%%%%%%%%%%%%%%%%%%%%%%%%%%%%%%%%%%%%%%%%%%%%%%%%%%%%%%%%%%%%
%
%%%%%%%%%%%%%%%%%%%%%%%%%%%%%%%%%%%%%%%%%%%%%%%%%%%%%%%%%%%%%%%%%%%%%%%%%%%%%%
\newpage
\section{HOBBITD\_ALERT}

 hobbitd\_alert - hobbitd worker module for sending out alerts

 \subsection{SYNOPSIS}
\textbf{hobbitd\_channel --channel=page hobbitd\_alert [options]}


 
\subsection{DESCRIPTION}
 hobbitd\_alert is a worker module for hobbitd, and as such it is
 normally run via the \emph{hobbitd\_channel(8)} program. It receives
 hobbitd page- and ack-messages from the ``page'' channel via stdin,
 and uses these to send out alerts about failed and recovered hosts
 and services. 


  The operation of this module is controlled by the
  \emph{hobbit-alerts.cfg(5)} file. This file holds the definition of
  rules and recipients, that determine who gets alerts, how often, for
  what servers etc. 



 
\subsection{OPTIONS}
\begin{description}
\item[--config=FILENAME] Sets the filename for the \emph{hobbit-alerts.cfg}
 file. The default value is ``etc/hobbit-alerts.cfg'' below the Hobbit server directory. 

 

\item[--dump-config] Dumps the configuration after parsing it. May be
  useful to track down problems with configuration file errors. 


 

\item[--checkpoint-file=FILENAME] File where the current state of the
  hobbitd\_alert module is saved. When starting up, hobbitd\_alert
  will also read this file to restore the previous state. 


 

\item[--checkpoint-interval=N] Defines how often (in seconds) the checkpoint-file is saved. 

 

\item[--cfid] If this option is present, alert messages will include a
  line with ``cfid:N'' where N is the linenumber in the
  hobbit-alerts.cfg file that caused this message to be sent. This can
  be useful to track down problems with duplicate alerts. 


 

\item[--test HOST SERVICE [options]] Shows which alert rules matches
  the given HOST/SERVICE combination. Useful to debug configuration
  problems, and see what rules are used for an alert. 


  The possible options are:  
\textbf{--color=COLORNAME}
 The COLORNAME parameter is the color of the alert: red, yellow or purple.  
\textbf{--duration=SECONDS}
 The SECONDS parameter is the duration of the alert in seconds.  
\textbf{--group=GROUPNAME}
 The GROUPNAME paramater is a groupid string from the hobbit-clients.cfg file.  
\textbf{--time=TIMESTRING}
 The TIMESTRING parameter is the time-of-day for the alert, expressed
 as an absolute time in the epoch format (seconds since Jan 1
 1970). This is easily obtained with the GNU date utility using the
 ``+\%s'' output format. 



\item[--debug] Enable debugging output. 

 


\end{description}
\subsection{HOW HOBBIT DECIDES WHEN TO SEND ALERTS}
 The hobbitd\_alert module is responsible for sending out all
 alerts. When a status first goes to one of the ALERTCOLORS,
 hobbitd\_alert is notified of this change. It notes that the status
 is now in an alert state, and records the \textbf{timestamp} when
 this event started, and adds the alert to the list statuses that may
 potentially trigger one or more alert messages. 


  This list is then matched against the hobbit-alerts.cfg
  configuration. This happens at least once a minute, but may happen
  more often. E.g. when status first goes into an alert state, this
  will always trigger the matching to happen. 



  When scanning the configuration, hobbitd\_alert looks at all of the
  configuration rules. It also checks the DURATION setting against how
  long time has elapsed since the event started - i.e. against the
  timestamp logged when hobbitd\_alert first heard of this event. 



  When an alert recipient is found, the alert is sent and it is
  recorded when this recipient is due for his next alert message,
  based on the REPEAT setting defined for this recipient. The next
  time hobbitd\_alert scans the configuration for what alerts to send,
  it will still find this recipient because all of the configuration
  rules are fulfilled, but an alert message will not be generated
  until the repeat interval has elapsed. 



  It can happen that a status first goes yellow and triggers an alert,
  and later it goes red - e.g. a disk filling up. In that case,
  hobbitd\_alert clears the internal timer for when the next (repeat)
  alert is due for all recipients. You generally want to be told when
  something that has been in a warning state becomes critical, so in
  that case the REPEAT setting is ignored and the alert is sent. This
  only happens the first time such a change occurs - if the status
  switches between yellow and red multiple times, only the first
  transition from yellow-$>$red causes this override. 



  When an status recovers, a recovery message may be sent - depending
  on the configuration - and then hobbitd\_alert forgets everything
  about this status. So the next time it goes into an alert state, the
  entire process starts all over again. 



 


 
\subsection{ENVIRONMENT}
\begin{description}
\item[MAIL] The first part of a command line used to send out an
  e-mail with a subject, typically set to ``/usr/bin/mail -s''
  . hobbitd\_alert will add the subject and the mail recipients to
  form the command line used for sending out email alerts. 


 

\item[MAILC] The first part of a command line used to send out an
  e-mail without a subject. Typically this will be
  ``/usr/bin/mail''. hobbitd\_alert will add the mail recipients to
  form the command line used for sending out email alerts. 


 
\end{description}
\subsection{FILES}
\begin{description}
\item[~hobbit/server/etc/hobbit-alerts.cfg]


\end{description}
\subsection{SEE ALSO}
hobbit-alerts.cfg(5), hobbitd(8), hobbitd\_channel(8), hobbit(7) 

 
%%%%%%%%%%%%%%%%%%%%%%%%%%%%%%%%%%%%%%%%%%%%%%%%%%%%%%%%%%%%%%%%%%%%%%%%%%%%%%
%
%%%%%%%%%%%%%%%%%%%%%%%%%%%%%%%%%%%%%%%%%%%%%%%%%%%%%%%%%%%%%%%%%%%%%%%%%%%%%%
\newpage
\section{BBHOSTSHOW}
 \index{bbhostshow} - output the full bb-hosts file 

\subsection{SYNOPSIS}
\textbf{bbhostshow [--bbdisp] [--bbnet] [filename]}


 
\subsection{DESCRIPTION}
\emph{bbhostshow(1)} dumps the full bb-hosts file to stdout. It
follows ``include'' tags in the bb-hosts files, and prints the full
contents as seen by the \emph{bbgen(1)} and \emph{bbtest-net(1)} utilities. 

  If no filename is given, bbhostshow displays the file pointed to by
  the BBHOSTS environment variable. 



 
\subsection{OPTIONS}
\begin{description}
\item[--bbdisp] Show the bb-hosts file following include statements as
  a BBDISPLAY server would. 


 

\item[--bbnet] Show the bb-hosts file following include statements as
  a BBNET server would. 

\end{description}
\subsection{ENVIRONMENT VARIABLES}
\begin{description}
\item[BBHOSTS] Filename for the \emph{bb-hosts(5)}
 file. 

 


\end{description}
\subsection{SEE ALSO}
bb-hosts(5), hobbitserver.cfg(5) 

  
%%%%%%%%%%%%%%%%%%%%%%%%%%%%%%%%%%%%%%%%%%%%%%%%%%%%%%%%%%%%%%%%%%%%%%%%%%%%%%
%
%%%%%%%%%%%%%%%%%%%%%%%%%%%%%%%%%%%%%%%%%%%%%%%%%%%%%%%%%%%%%%%%%%%%%%%%%%%%%%
\newpage
\section{HOBBITD\_CAPTURE}

 hobbitd\_capture - catch selected messages from a hobbitd channel

\subsection{SYNOPSIS}
\textbf{hobbitd\_channel --channel=status hobbitd\_capture [options]}


 
\subsection{DESCRIPTION}
 hobbitd\_capture is a worker module for hobbitd, and as such it is
 normally run via the \emph{hobbitd\_channel(8)} program. It receives
 messages from hobbitd via stdin and filters them to select messages
 based on the hostname, testname or color of the status. By default
 the resulting messages are printed on stdout, but they can also be
 fed into a command for further processing. 


  hobbitd\_capture supports the \textbf{status}
, \textbf{data}
 and \textbf{hostdata}
 channels. 


 
\subsection{OPTIONS}
\begin{description}
\item[--hosts=PATTERN] Select messages only from hosts matching
  PATTERN (regular expression). 


 

\item[--exhosts=PATTERN] Exclude messages from hosts matching
  PATTERN. If used with the --hosts option, then the hostname must
  match the --hosts pattern, but NOT the --exhosts pattern. 


 

\item[--tests=PATTERN] Select messages only from tests matching PATTERN (regular expression). 

 

\item[--extests=PATTERN] Exclude messages from tests matching
  PATTERN. If used with the --tests option, then the testname must
  match the --tests pattern, but NOT the --extests pattern. 


 

\item[--colors=COLOR[,color]] Select messages based on the color of
  the status message. Multiple colors can be listed, separated by
  comma. Default: Accept all colors. 


 

\item[--batch-command=COMMAND] Instead of printing the messages to
  stdout, feed them to COMMAND on stdin. COMMAND can be any command
  which accepts the mssage on standard input. 


 

\item[--batch-timeout=SECONDS] Collect messages until no messages have arrived in SECONDS seconds, before sending them to the --batch-command COMMAND. 

 

\item[--debug] Enable debugging output. 

 


\end{description}
\subsection{SEE ALSO}
hobbitd\_channel(8), hobbitd(8), hobbit(7) 

 
%%%%%%%%%%%%%%%%%%%%%%%%%%%%%%%%%%%%%%%%%%%%%%%%%%%%%%%%%%%%%%%%%%%%%%%%%%%%%%
%
%%%%%%%%%%%%%%%%%%%%%%%%%%%%%%%%%%%%%%%%%%%%%%%%%%%%%%%%%%%%%%%%%%%%%%%%%%%%%%
\newpage
\section{HOBBIT-STATUSREPORT.CGI}

 hobbit-statusreport.cgi - CGI program to report a status for a group
 of servers
 \subsection{SYNOPSIS}
\textbf{hobbit-statusreport.cgi --column=COLUMNNAME [options]}


 
\subsection{DESCRIPTION}
\textbf{hobbit-statusreport.cgi} is a CGI tool to generate a simple
HTML report showing the current status of a single column for a group
of Hobbit hosts. 


  E.g. You can use this report to get an overview of all of the SSL certificates that are about to expire. 


  The generated webpage is a simple HTML table, suitable for copying into other documents or e-mail. 


  hobbit-statusreport.cgi runs as a CGI program, invoked by your webserver. It is normally run via a wrapper shell-script in the CGI directory for Hobbit. 


 
\subsection{EXAMPLES}
 The Hobbit installation includes two web report scripts using this
 CGI tool: The \textbf{hobbit-certreport.sh} script generates a list
 of SSL server certificates that are yellow or red (i.e. they will
 expire soon); and the \textbf{hobbit-nongreen.sh} script generates a
 report of all statuses that are currently non-green. These can be
 accessed from a web browser through a URL referencing the script in
 the Hobbit CGI directory (e.g. ``/hobbit-cgi/hobbit-nongreen.sh''). 


 
\subsection{OPTIONS}
\begin{description}
\item[--column=COLUMNNAME] Report the status of the COLUMNNAME column. 

 

\item[--all] Report the status for all hosts known to Hobbit. By
  default, this tool reports only on the hosts found on the current
  page from where the CGI was invoked (by looking at the ``pagepath''
  cookie). 


 

\item[--filter=CRITERIA] Only report on statuses that match the
  CRITERIA setting. See the \emph{bb(1)} man-page - in the
  ``hobbitdboard'' command description - for details about specifying
  filters. 


 

\item[--heading=HTML] Defines the webpage heading - i.e. the ``title'' tag in the generated HTML code. 

 

\item[--show-column] Include the column name in the display. 

 

\item[--show-colors] Show the status color on the generated webpage. The default is to not show the status color. 

 

\item[--no-colors] Do not include text showing the current color of each status in the report. This is the default. 

 

\item[--show-summary] Show only a summary of the important lines in
  the status message. By default, the entire status message appears in
  the generated HTML code. This option causes the first non-blank line
  of the status message to be shown, and also any lines beginning with
  ``\&COLOR'' which is used by many status messages to point out lines
  of interest (non-green lines only, though). 


 

\item[--show-message] Show the entire message on the webpage. This is the default. 

 

\item[--link] Include HTML links to the host ``info'' page, and the status page. 

 

\item[--embedded] Only generate the HTML table, not a full
  webpage. This can be used to embed the status report into an
  external webpage. 


 

\item[--env=FILENAME] Load the environment from FILENAME before executing the CGI. 

 

\item[--area=NAME] Load environment variables for a specific area. NB:
  if used, this option must appear before any --env=FILENAME option. 


 


\end{description}
\subsection{SEE ALSO}
hobbit(7) 

%%%%%%%%%%%%%%%%%%%%%%%%%%%%%%%%%%%%%%%%%%%%%%%%%%%%%%%%%%%%%%%%%%%%%%%%%%%%%%
%
%%%%%%%%%%%%%%%%%%%%%%%%%%%%%%%%%%%%%%%%%%%%%%%%%%%%%%%%%%%%%%%%%%%%%%%%%%%%%%
%\include{bbmessage.cgi.8}
\newpage
\section{BBMESSAGE.CGI}

 bbmessage.cgi - CGI utility used for proxying Hobbit data over HTTP

 \subsection{SYNOPSIS}
\textbf{bbmessage.cgi}


 
\subsection{DESCRIPTION}
\emph{bbmessage.cgi(8)}
 is the server-side utility receiving Hobbit messages sent by the
 \emph{bb(1)} utility over an HTTP transport. The \textbf{bb} utility
 normally sends data over a dedicated TCP protocol, but it may use
 HTTP to go through proxies or through restrictive firewalls. In that
 case, the webserver must have this CGI utility installed, which takes
 care of receiving the message via HTTP, and forwards it to a local
 Hobbit server through the normal Hobbit transport. 


  The CGI expects to be invoked from an HTTP ``POST'' request, with
  the POST-data being the status-message. \textbf{bbmessage.cgi}
  simply collects all of the POST data, and send it off as a message
  to the Hobbit daemon running on IP 127.0.0.1. This destination IP
  currently cannot be changed. 



  The CGI will return any output provided by the Hobbit daemon back to
  the requestor as the response to the HTTP POST, so this allows for
  all normal Hobbit commands to work. 



 
\subsection{SECURITY}
\textbf{bbmesage.cgi} will only send data to a Hobbit server through
the loopback interface, i.e. IP-address 127.0.0.1. 


  Access to the CGI should be restricted through webserver access
  controls, since the CGI provides no authentication at all to
  validate incoming messages. 



  If possible, consider using the \emph{bbproxy(8)}
 utility instead for native proxying of Hobbit data between networks. 


 
\subsection{SEE ALSO}
bb(1), bbproxy(8), hobbit(7) 


%%%%%%%%%%%%%%%%%%%%%%%%%%%%%%%%%%%%%%%%%%%%%%%%%%%%%%%%%%%%%%%%%%%%%%%%%%%%%%
%
%%%%%%%%%%%%%%%%%%%%%%%%%%%%%%%%%%%%%%%%%%%%%%%%%%%%%%%%%%%%%%%%%%%%%%%%%%%%%%
\newpage
\section{BB-WEBPAGE}
 bb-webpage - Utility to show a webpage using header and footer \
\subsection{SYNOPSIS}
\textbf{bb-webpage [options]}


 
\subsection{DESCRIPTION}
\textbf{bb-webpage} is a tool to generate a webpage in the Hobbit
style, with a standard header- and footer as well as a Hobbit
background. The data to present on the webpage, apart from the header
and footer, are passed to bb-webpage in stdin. The generated webpage
is printed to stdout. 


 
\subsection{OPTIONS}
\begin{description}
\item[--env=FILENAME] Loads the environment defined in FILENAME before executing the CGI script. 

 

\item[--hffile=PREFIX] Use the header- and footer-files in
  \$BBHOME/web/PREFIX\_header and PREFIX\_footer. If not specified,
  bb\_header and bb\_footer are used. 


 

\item[--color=COLOR] Set the background color of the generated webpage to COLOR. Default: Blue 

 

\item[--debug] Enable debugging output. 

 
\end{description}
\subsection{SEE ALSO}
hobbit(7) 

 





