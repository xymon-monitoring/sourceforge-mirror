%%%%%%%%%%%%%%%%%%%%%%%%%%%%%%%%%%%%%%%%%%%%%%%%%%%%%%%%%%%%%%%%%%%%%%%%%%%%%%
%
%%%%%%%%%%%%%%%%%%%%%%%%%%%%%%%%%%%%%%%%%%%%%%%%%%%%%%%%%%%%%%%%%%%%%%%%%%%%%%

\chapter{Hobbit/BB Messages Protocols}

%%%%%%%%%%%%%%%%%%%%%%%%%%%%%%%%%%%%%%%%%%%%%%%%%%%%%%%%%%%%%%%%%%%%%%%%%%%%%%
%
%%%%%%%%%%%%%%%%%%%%%%%%%%%%%%%%%%%%%%%%%%%%%%%%%%%%%%%%%%%%%%%%%%%%%%%%%%%%%%
\section{BNF}

Extended Backus.Naur Form (\url{http://en.wikipedia.org/wiki/Backus-Naur_form} BNF) is a metasyntax used to express context-free grammars: that is, a formal way to describe formal languages. We use BNF to describe the formal language in used between hobbit client and server.

\subsection{Hobbit messages syntax}

\begin{verbatim}

<hb-messages> ::= <status_msg> <notify_msg> <data_msg> <enable_msg> <query_msg> 
                <config_msg> <drop_msg> <rename_msg> <hobbidlog_msg> <hobbidxlog_msg> 
                <hobbitdboard_msg> <download_msg> <client_msg> <clientlog_msg> <ping_msg> 
                <pullclient_msg> <ghostlist_msg> <schedule_msg>

<drop_msg>   ::= <drop HOSTNAME> <EOL> | <drop HOSTNAME TESTNAME> <EOL>
<HOSTNAME>   ::= <a..z> 
<TESTNAME>   ::= <cpu|trends|conn|info|> 
<status_msg> ::= <status[+LIFETIME][/group:GROUP] HOSTNAME.TESTNAME COLOR |
                 <additional text> ]> <EOL>
<notify_msg> ::= <notify HOSTNAME.TESTNAME <message text> ] >
<data_msg>   ::= <data HOSTNAME.DATANAME <newline> <additional text> >
<enable_msg> ::= [enable HOSTNAME.TESTNAME]
<query_msg>  ::= [query HOSTNAME.TESTNAME]
<config_msg> ::= [config FILENAME]
<drop_msg>   ::= [drop HOSTNAME]
<rename_msg> ::= [rename OLDHOSTNAME NEWHOSTNAME]
<hobbidlog_msg>    ::= 
<hobbidxlog_msg>   ::= [hobbitdxlog HOSTNAME.TESTNAME]
<hobbitdboard_msg> ::= [hobbitdboard [CRITERIA] [fields=FIELDLIST]]
<download_msg>     ::= [download FILENAME] <EOL>
<client_msg>       ::= [client HOSTNAME.OSTYPE [HOSTCLASS]]
<clientlog_msg>    ::= [clientlog HOSTNAME [section=SECTIONNAME[,SECTIONNAME...]]]
<ping_msg>         ::= [ping]  <EOL>
<pullclient_msg>   ::= [pullclient] <EOL>
<ghostlist_msg>    ::= [ghostlist] <EOL>
<schedule_msg>     ::= [schedule [TIMESTAMP COMMAND]]  <EOL>
<SECTIONNAME>      ::= [cpu|la| others sectionname]

\end{verbatim}


\subsection{Big Brother messages syntax}

\begin{verbatim}

<bb-messages> ::= <status_msg> <notify_msg> <data_msg> <enable_msg> <query_msg> 
                <config_msg> <drop_msg> <rename_msg> <hobbidlog_msg> <hobbidxlog_msg> 
                <hobbitdboard_msg> <download_msg> <client_msg> <clientlog_msg> <ping_msg> 
                <pullclient_msg> <ghostlist_msg> <schedule_msg>

<drop_msg>   ::= <drop HOSTNAME> <EOL> | <drop HOSTNAME TESTNAME> <EOL>
<HOSTNAME>   ::= <a..z> 
<TESTNAME>   ::= <cpu|trends|conn|info|> 
<status_msg> ::= <status[+LIFETIME][/group:GROUP] HOSTNAME.TESTNAME COLOR |
                 <additional text> ]> <EOL>
<notify_msg> ::= <notify HOSTNAME.TESTNAME <message text> ] >
<data_msg>   ::= <data HOSTNAME.DATANAME <newline> <additional text> >
<enable_msg> ::= [enable HOSTNAME.TESTNAME]
<query_msg>  ::= [query HOSTNAME.TESTNAME]
<config_msg> ::= [config FILENAME]
<drop_msg>   ::= [drop HOSTNAME]
<rename_msg> ::= [rename OLDHOSTNAME NEWHOSTNAME]
<hobbidlog_msg>    ::= 
<hobbidxlog_msg>   ::= [hobbitdxlog HOSTNAME.TESTNAME]
<hobbitdboard_msg> ::= [hobbitdboard [CRITERIA] [fields=FIELDLIST]]
<download_msg>     ::= [download FILENAME] <EOL>
<client_msg>       ::= [client HOSTNAME.OSTYPE [HOSTCLASS]]
<clientlog_msg>    ::= [clientlog HOSTNAME [section=SECTIONNAME[,SECTIONNAME...]]]
<ping_msg>         ::= [ping]  <EOL>
<pullclient_msg>   ::= [pullclient] <EOL>
<ghostlist_msg>    ::= [ghostlist] <EOL>
<schedule_msg>     ::= [schedule [TIMESTAMP COMMAND]]  <EOL>
<SECTIONNAME>      ::= [cpu|la| others sectionname]

\end{verbatim}

%%%%%%%%%%%%%%%%%%%%%%%%%%%%%%%%%%%%%%%%%%%%%%%%%%%%%%%%%%%%%%%%%%%%%%%%%%%%%%
%
%%%%%%%%%%%%%%%%%%%%%%%%%%%%%%%%%%%%%%%%%%%%%%%%%%%%%%%%%%%%%%%%%%%%%%%%%%%%%%
\newpage
\section{bb - Hobbit client communication program}

\subsection{SYNOPSIS}
\textbf{\motohbcmd{bb} [options] RECIPIENT message}

\subsection{DESCRIPTION}
\emph{bb(1)} is the client program used to communicate with a Hobbit
server. It is frequently used by Hobbit client systems to send in
status messages and pager alerts on local tests. 


In Hobbit, the bb program is also used for administrative purposes,
e.g. to rename or delete hosts, or to disable hosts that are down for
longer periods of time. 

 
\subsection{OPTIONS AND PARAMETERS}
\begin{description}

\item[--debug] Enable debugging. This prints out details about how the
  connection to the BBDISPLAY server is being established. 

\item[--proxy=\url{http://PROXYSERVER:PROXYPORT/}] When sending the
  status messages via HTTP, use this server as an HTTP proxy instead
  of connecting directly to the BBDISPLAY server. 

\item[--timeout=N] Specifies the timeout for connecting to the Hobbit
  server, in seconds. The default is 5 seconds. 

\item[RECIPIENT] The \textbf{RECIPIENT}
 parameter defines which server receives the message. If RECIPIENT is
 given as ``0.0.0.0'', then the message is sent to all of the servers
 listed in the BBDISPLAYS or BBPAGERS environment variable (for
 ``status'' and ``page'' messages, respectively). 

Usually, a client will use ``\$BBDISP'' for the \textbf{RECIPIENT}
parameter, as this is defined for the client scripts to automatically
contain the correct value. 


The \textbf{RECIPIENT} parameter may be a URL for a webserver that
has the bbmessage.cgi or similar script installed. This tunnels the
Hobbit messages to the BBDISPLAY server using standard HTTP
protocol. The \emph{bbmessage.cgi(8)} CGI tool (included in Hobbit)
must be installed on the webserver for the HTTP transport to work.  


\item[MESSAGE] The \textbf{message} parameter is the message to be
  sent across to the Hobbit server. Messages must be enclosed in
  quotes, but by doing so they can span multiple lines. The maximum
  size of a message is defined by the maximum allowed length of your
  shell's commandline, and is typically  8-32 KB. 

If you need to send longer status messages, you can specify ``@'' as
the message: bb will then read the status message from its stdin. 


\end{description}

\subsection{HOBBIT MESSAGE SYNTAX}


This section lists the most commonly used messages in the Hobbit protocol. 


\begin{description}

\item[status[+LIFETIME][/group:GROUP] HOSTNAME.TESTNAME COLOR
  $<$additional text$>$] This sends in a status message for a single
  test (column) on a single host. TESTNAME is the name of the column
  where this test will show up; any name is valid except that using
  dots in the testname will not work. COLOR must be one of the valid
  colors: ``green'', ``yellow'', ``red'' or ``clear''. The colors
  ``blue'' and ``purple'' - although valid colors - should not be sent
  in a status-message, as these are handled specially by the Hobbit
  server.  

The ``additional text'' normally includes a local timestamp and a
summary of the test result on the first line. Any lines following the
first one are free-form, and can include any information that may be
useful to diagnose the problem being reported.  

The LIFETIME defines how long this status is valid after being
received by the Hobbit server. The default is 30 minutes, but you can
set any period you like. E.g. for a custom test that runs once an
hour, you will want to set this to at least 60 minutes - otherwise the
status will go purple after 30 minutes. It is a good idea to set the
LIFETIME to sligtly more than the interval between your tests, to
allow for variations in the time it takes your test to complete. The
LIFETIME is in minutes, unless you add an ``h'' (hours), ``d'' (days)
or ``w'' (weeks) immediately after the number, e.g. ``status+5h'' for
a status that is valid for 5 hours.  


The GROUP option is used to direct alerts from the status to a
specific group. It is currently used for status generated from the
Hobbit clients' data, e.g. to direct alerts for a ``procs'' status to
different people, depending on exactly which process is down. 


 

\item[notify HOSTNAME.TESTNAME $<$message text$>$] This triggers an
 informational message to be sent to those who receive alerts for this
 HOSTNAME+TESTNAME combination, according to the rules defined in
 \emph{hobbit-alerts.cfg(5)}  This is used by the
 \emph{hobbit-enadis.cgi(1)} tool to notify people about hosts being
 disabled or enabled, but can also serve as a general way of notifying
 server administrators. 


 

\item[data HOSTNAME.DATANAME$<$newline$>$$<$additional text$>$] The
 ``data'' message allows tools to send data about a host, without it
 appearing as a column on the Hobbit webpages. This is used e.g. to
 report statistics about a host, e.g. vmstat data which does not in
 itself represent something that has a red, yellow or green
 identity. It is used by RRD bottom-feeder modules, among others. In
 Hobbit, data messages are by default processed only by the
 \emph{hobbitd\_rrd(8)} module. If you want to handle data-messages by
 an external application, you may want to enable the
 \emph{hobbitd\_filestore(8)}  module for data-messages, to store
 data-messages in a format compatible with how the Big Brother daemon
 does. 


%\item[disable HOSTNAME.TESTNAME DURATION $<$additional text$>$]
 Disables a specific test for DURATION minutes. This will cause the
 status of this test to be listed as ``blue'' on the BBDISPLAY server,
 and no alerts for this host/test will be generated. If DURATION is
 given as a number followed by s/m/h/d, it is interpreted as being in
 seconds/minutes/hours/days respectively. \textbf{To} 

%disable\textbf{all}
%tests\textbf{for}
%a\textbf{host,}
%use\textbf{an}
%asterisk\textbf{*}
%for\textbf{TESTNAME.}


\item[enable HOSTNAME.TESTNAME] Re-enables a test that had been disabled. 

 

\item[query HOSTNAME.TESTNAME] Query the BBDISPLAY server for the
  latest status reported for this particular test. If the host/test
  status is known, the response is the first line of the status report
  - the current color will be the first word on the line. Additional
  lines of text that might be present on the status-message cannot be
  retrieved.  

 This allows any Hobbit client to determine the status of a particular
 test, whether it is one pertaining to the host where the client is
 running, some other host, or perhaps the result of a combined test
 from multiple hosts managed by \emph{bbcombotest(1)} This will
 typically be useful to Hobbit client extension scripts, that need to
 determine the status of other hosts e.g. to decide if an automatic
 recovery action should be initiated. 


 

\item[config FILENAME] Retrieve one of the Hobbit configuration files
  from the server. This command allows a client to pull files from the
  \$BBHOME/etc/ directory on the server, allowing for semi-automatic
  updates of the client configuration. Since the configuration files
  are designed to have a common file for the configuration of all
  hosts in the system - and this is in fact the recommended way of
  configuring your clients - this makes it easier to keep the
  configuration files synchronized. 


 

\item[drop HOSTNAME] Removes all data stored about the host
  HOSTNAME. It is assumed that you have already deleted the host from
  the bb-hosts configuration file. 


 

\item[drop HOSTNAME TESTNAME] Remove data about a single test (column). 

 

\item[rename OLDHOSTNAME NEWHOSTNAME] Rename all data for a host that
  changes its name. You should do this before changing the hostname in
  the bb-hosts configuration file. 


 

\item[rename HOSTNAME OLDTESTNAME NEWTESTNAME] Rename data about a single test (column). 

 

\item[hobbitdlog HOSTNAME.TESTNAME] Retrieve the Hobbit status-log for
  a single test. The first line of the response contain a series of
  fields separated by a pipe-sign: 


 \textbf{hostname}
 The name of the host 


 \textbf{testname}
 The name of the test 


 \textbf{color}
 Status color (green, yellow, red, blue, clear, purple) 


 \textbf{testflags}
 For network tests, the flags indicating details about the test (used by bbgen). 


 \textbf{lastchange}
 Unix timestamp when the status color last changed. 


 \textbf{logtime}
 Unix timestamp when the log message was received. 


 \textbf{validtime}
 Unix timestamp when the log message is no longer valid (it goes purple at this time). 


 \textbf{acktime}
 -1, or Unix timestamp when an active acknowledgement expires. 


 \textbf{disabletime}
 -1, or Unix timestamp when the status is no longer disabled. 


 \textbf{sender}
 IP-address where the status was received from. 


 \textbf{cookie}
 -1, or the cookie value used to acknowledge an alert. 


 \textbf{ackmsg}
 Empty, or the acknowledgment message sent when the status was
 acknowledged. Newline, pipe-signs and backslashes are escaped by with
 a backslash in C-style. 



 \textbf{dismsg}
 Empty, or the message sent when the status was disabled. Newline,
 pipe-signs and backslashes are escaped by with a backslash in
 C-style. 



  After the first line comes the full status log in plain text format. 


 

\item[hobbitdxlog HOSTNAME.TESTNAME] Retrieves an XML-string with the
  status log as for the ``hobbitdlog'' command. 


\item[hobbitdboard [CRITERIA] [fields=FIELDLIST]] Retrieves a summary
  of the status of all known tests available to the Hobbit daemon. 


  By default - if no CRITERIA is provided - it returns one line for
  all status messages that are found in Hobbit. You can filter the
  response by selecting a page, a host, a test or a color - wildcards
  are not supported, so you can pick only one page, host, test or
  color. 

 \textbf{page=PAGEPATH}
 Include only tests from hosts found on the PAGEPATH page in the bb-hosts file. 


 \textbf{host=HOSTNAME}
 Include only tests from the host HOSTNAME 


 \textbf{test=TESTNAME}
 Include only tests with the testname TESTNAME 


 \textbf{color=COLORNAME}
 Include only tests where the status color is COLORNAME 


  You can filter on e.g. both a hostname and a testname. 


  The response is one line for each status that matches the CRITERIA,
  or all statuses if no criteria is specified. The line is composed of
  a number of fields, separated by a pipe-sign. You can select what
  fields to retrieve by listing them in the FIELDLIST. The following
  fields are available: 


 \textbf{hostname}
 The name of the host 


 \textbf{testname}
 The name of the test 


 \textbf{color}
 Status color (green, yellow, red, blue, clear, purple) 


 \textbf{flags}
 For network tests, the flags indicating details about the test (used by bbgen). 


 \textbf{lastchange}
 Unix timestamp when the status color last changed. 


 \textbf{logtime}
 Unix timestamp when the log message was received. 


 \textbf{validtime}
 Unix timestamp when the log message is no longer valid (it goes purple at this time). 


 \textbf{acktime}
 -1, or Unix timestamp when an active acknowledgement expires. 


 \textbf{disabletime}
 -1, or Unix timestamp when the status is no longer disabled. 


 \textbf{sender}
 IP-address where the status was received from. 


 \textbf{cookie}
 -1, or the cookie value used to acknowledge an alert. 


 \textbf{line1}
 First line of status log. 


 \textbf{ackmsg}
 Empty (if no acknowledge is active), or the text of the acknowledge message. 


 \textbf{dismsg}
 Empty (if the status is currently enabled), or the text of the disable message. 


 \textbf{msg}
 The full text of the current status message. 

  The ackmsg, dismsg and msg fields have certain characters encoded:
  Newline is ``$\backslash$n'', TAB is ``$\backslash$t'', carriage
  return is ``$\backslash$r'', a pipe-sign is ``$\backslash$p'', and a
  backslash is ``$\backslash$$\backslash$''. 



  If the ``fields'' parameter is omitted, a default set of
  hostname,testname,color,flags,lastchange,logtime,validtime,acktime,disabletime,sender,cookie,line1
  is used. 

\item[hobbitdxboard] Retrieves an XML-string with the summary of all
  status logs as for the ``hobbitdboard'' command. 



 

\item[download FILENAME] Download a file from the Hobbit servers' download directory. 

 

\item[client HOSTNAME.OSTYPE [HOSTCLASS]] Used to send a ``client''
  message to the Hobbit server. Client messages are generated by the
  Hobbit client; when sent to the Hobbit server they are matched
  against the rules in the \emph{hobbit-clients.cfg(5)} configuration
  file, and status messages are generated for the client-side tests. 


 

\item[clientlog HOSTNAME [section=SECTIONNAME[,SECTIONNAME...]]]
  Retrieves the current raw client message last sent by HOSTNAME. The
  optional ``section'' filter is used to select specific sections of
  the client data. 



\item[ping] Attempts to contact the Hobbit server. If successful, the
  Hobbit server version ID is reported. 


 

\item[pullclient] This message is used when fetching client data via
  the ``pull'' mechanism implemented by \emph{hobbitfetch(8)} and
  \emph{msgcache(8)} for clients that cannot connect directly to the
  Hobbit server. 



\item[ghostlist] Report a list of \textbf{ghost} clients seen by the
  Hobbit server. Ghosts are systems that report data to the Hobbit
  server, but are not listed in the bb-hosts file. 

 

\item[schedule [TIMESTAMP COMMAND]] Schedules an command sent to the
  Hobbit server for execution at a later time. E.g. used to schedule
  disabling of a host or service at sometime in the future. COMMAND is
  a complete Hobbit command such as the ones listed above. TIMESTAMP
  is the Unix epoch time when the command will be executed.  

 If no parameters are given, the currently scheduled tasks are listed
 in the response. The response is one line per scheduled command, with
 the job-id, the time when the command will be executed, the IP
 address from which this was sent, and the full command string.   To
 cancel an previously scheduled command, \textbf{``schedule cancel
 JOBID''} can be used. JOBID is a number provided as the first item in
 the output from the schedule list. 


 


\end{description}

\subsection{EXAMPLE}


 Send a normal status message to the BBDISPLAY server, using the
 standard Hobbit protocol on TCP port 1984:  

 
\$\$BB\$BBDISP''statuswww.foo.com.httpgreen`date`WebOK'' 


  Send the same status message, but using HTTP protocol via the webservers' bbmessage.cgi script:  
 
\$\$BB\url{http://bb.foo.com/cgi-bin/bbmessage.cgi}''statuswww.foo.com.httpgreen`date`WebOK'' 


  Use ``query'' message to determine the color of the ``www'' test, and restart Apache if it is red:  



\$WWW=`\$BB\$BBDISP''querywww.foo.com.www``|awk'\{print\$1\}'`  \$if[''\$WWW''=''red''];then/etc/init.d/apacherestart;fi 


  Use ``config'' message to update the local bb-dftab file (but only if we get a response):  


  
\$\$BB\$BBDISP''configbb-dftab''$>$/tmp/bb-dftab.new  
\$if[-s/tmp/bb-dftab.new];then  
mv/tmp/bb-dftab.new\$BBHOME/etc/bb-dftab  
fi 


  Send a very large status message that has been built in the file
  ``statusmsg.txt''. Instead of providing it on the command line, pass
  it via stdin to the bb command: 



  
\$catstatusmsg.txt|\$BB\$BBDISP''@'' 


 
\subsection{NOTES}
 This man-page describes the bb client program provided as part of
 Hobbit . This implementation provides features not present in the
 standard Big Brother bb client - specifically, the support for
 sending messages over HTTP, and many commands such as ``query'' and
 ``config'' are not part of the bb client shipped with Big Brother. 


 
\subsection{SEE ALSO}

bbcombotest(1), bb-hosts(5), hobbitserver.cfg(5), hobbit(7) 
  

