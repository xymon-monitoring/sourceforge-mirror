%%%%%%%%%%%%%%%%%%%%%%%%%%%%%%%%%%%%%%%%%%%%%%%%%%%%%%%%%%%%%%%%%%%%%%%%%%%%%%
%
%%%%%%%%%%%%%%%%%%%%%%%%%%%%%%%%%%%%%%%%%%%%%%%%%%%%%%%%%%%%%%%%%%%%%%%%%%%%%%
\chapter{Miscellanceours programs}

%%%%%%%%%%%%%%%%%%%%%%%%%%%%%%%%%%%%%%%%%%%%%%%%%%%%%%%%%%%%%%%%%%%%%%%%%%%%%%
%
%%%%%%%%%%%%%%%%%%%%%%%%%%%%%%%%%%%%%%%%%%%%%%%%%%%%%%%%%%%%%%%%%%%%%%%%%%%%%%
\newpage
\section{HOBBITLAUNCH}
 xymonlaunch - Master program to launch other xymon programs 

 
\subsection{SYNOPSIS}
\textbf{xymonlaunch [options]}


 
\subsection{DESCRIPTION}
\emph{xymonlaunch(8)} is the main program that controls the execution
and scheduling of all of the components in the xymon system. 


  xymonlaunch allows the administrator to add, remove or change the
  set of xymon applications and extensions without restarting Xymon
  - xymonlaunch will automatically notice any changes in the set of
  tasks, and change the scheduling of activities accordingly. 



  xymonlaunch also allows the administrator to setup specific
  logfiles for each component of the xymon system, instead of getting
  output from all components logged to a single file. 



 
\subsection{OPTIONS}
\begin{description}
\item[--env=FILENAME] Loads the environment from FILENAME before
  starting other tools. The environment defined by FILENAME is the
  default, it can be overridden by the ENVFILE option in
  \emph{xymonlaunch.cfg(5)}



 

\item[--config=FILENAME] This option defines the file that
  xymonlaunch scans for tasks it must launch. A description of this
  file is in \emph{xymonlaunch.cfg(5)}

 The default tasklist is /etc/xymonlaunch.cfg 

 

\item[--log=FILENAME] Defines the logfile where xymonlaunch logs
  information about failures to launch tasks and other data about the
  operation of xymonlaunch. Logs from individual tasks are defined in
  the xymonlaunch.cfg file. By default this is logged to stdout. 


 

\item[--pidfile=FILENAME] Filename which xymonlaunch saves its own process-ID to. Commonly used by automated start/stop scripts. 

 

\item[--verbose] Logs the launch of all tasks to the logfile. Note that the logfile may become quite large if you enable this. 

 

\item[--dump] Just dump the contents of the xymonlaunch.cfg file after parsing it. Used for debugging. 

 

\item[--debug] Enable debugging output while running. 

 

\item[--no-daemon] xymonlaunch normally detaches from the controlling
  tty and runs as a background task. This option keeps it running in
  the foreground. 


 

\end{description}

\subsection{STARTING TASKS}  xymonlaunch will read the configuration
 file and start all of the tasks listed there. 


  If a task completes abnormally (i.e. terminated by a signal or with
  a non-zero exit status), then xymonlaunch will attempt to restart
  it 5 times. If it still will not run, then the task is disabled for
  10 minutes. This will be logged to the xymonlaunch logfile. 


  If the configuration file changes, xymonlaunch will re-read it and
  notice any changes. If a running task was removed from the
  configuration, then the task is stopped. If a new task was added, it
  will be started. If the command used for a task changed, or it was
  given a new environment definition file, or the logfile was changed,
  then the task is stopped and restarted with the new definition. 



 
\subsection{SEE ALSO}
xymonlaunch.cfg(5), hobbit(7) 
