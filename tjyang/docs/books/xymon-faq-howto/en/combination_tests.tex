%%%%%%%%%%%%%%%%%%%%%%%%%%%%%%%%%%%%%%%%%%%%%%%%%%%%%%%%%%%%%%%%%%%%%%%%%%%%%%
%
%%%%%%%%%%%%%%%%%%%%%%%%%%%%%%%%%%%%%%%%%%%%%%%%%%%%%%%%%%%%%%%%%%%%%%%%%%%%%%
\chapter{Combination Tests}

\section{BBCOMBOTEST}
 bbcombotest - Hobbit combination test tool 

\subsection{SYNOPSIS}
\textbf{bbcombotest --help}
\textbf{bbcombotest --version}
\textbf{bbcombotest [--debug] [--quiet]}


 
\subsection{DESCRIPTION}
\textbf{bbcombotest} is a Hobbit extension script that runs on the
BBDISPLAY server. It combines the results of one or more of the normal
Hobbit test results into a combined test result, using standard
arithmetic og logical operators. 


The resulting tests are sent to the Hobbit display server as any
normal test - so all of the standard Hobbit functions (history,
statistics etc.) are available for the combined tests. 


The tool was born from the need to monitor systems with built-in
redundancy and automatic failover - e.g. load-balanced web
servers. But other uses are possible. 

 
\subsection{OPTIONS}

\begin{description}
\item[--quiet] Normally, the test status sent by bbcombotest includes
  information about the underlying test results used to determine the
  current value of the combined test. ``--quiet'' eliminates this
  information from the test status page. 

\item[--debug] Provide debugging output for use in troubleshooting
  problems with bbcombotest. 


\item[--no-update] Dont send any status messages - instead, the result
  of the combotests is simply dumped to stdout. Useful for debugging. 



\end{description}
\subsection{FILES}
\begin{description}

\item[\motoserverconfig{bbcombotest.cfg}] Configuration file for
  bbcombotest, where the combined tests are defined 

\item[\motoserverconfig{hobbitlaunch.cfg}] Configuration file controlling
  when bbcombotest is run. 



\end{description}
\subsection{SEE ALSO}
bbcombotest.cfg(5), bb-hosts(5), hobbitserver.cfg(5), hobbitlaunch.cfg(5) 

%%%%%%%%%%%%%%%%%%%%%%%%%%%%%%%%%%%%%%%%%%%%%%%%%%%%%%%%%%%%%%%%%%%%%%%%%%%%%%
%
%%%%%%%%%%%%%%%%%%%%%%%%%%%%%%%%%%%%%%%%%%%%%%%%%%%%%%%%%%%%%%%%%%%%%%%%%%%%%%
\newpage
\section{HOBBITGRAPH.CGI}
\subsection{NAME}
 hobbitgraph.cgi - CGI to generate Hobbit trend graphs 
\subsection{SYNOPSIS}
\textbf{hobbitgraph [options]}


 
\subsection{DESCRIPTION}
\textbf{hobbitgraph.cgi}
 is invoked as a CGI script via the hobbitgraph.sh CGI wrapper. 

  hobbitgraph.cgi is passed a QUERY\_STRING environment variable with the following parameters: 


 \textbf{host}
 Name of the host to generate a graph for 


 \textbf{service}
 Name of the service to generate a graph for 


 \textbf{disp}
 Display-name of the host, used on the generated graphs instead of hostname. 


 \textbf{graph}
 Can be ``hourly'', ``daily'', ``weekly'' or ``monthly'' to select the time period that the graph covers. 


 \textbf{first}
 Used to split multi-graphs into multiple graphs. This causes
 hobbitgraph.cgi to generate only the graphs starting with the
 ``first'th'' graph and continuing for ``count''. 


 \textbf{count}
 Number of graphs in a multi-graph. 


 \textbf{upper}
 Set the upper limit of the graph. See \emph{rrdgraph(1)}
 for a description of the ``-u'' option. 


 \textbf{lower}
 Set the lower limit of the graph. See \emph{rrdgraph(1)}
 for a description of the ``-l'' option. 


 \textbf{graph\_start}
 Set the starttime of the graph. This is used in zoom-mode. 


 \textbf{graph\_end}
 Set the end-time of the graph. This is used in zoom-mode. 


 \textbf{action=menu}
 Generate an HTML page with links to 4 graphs, representing the hourly, weekly, monthly and yearly graphs. Doesn't actually generate any graphs, only the HTML that links to the graphs. 


 \textbf{action=selzoom}
 Generate an HTML page with link to single graph, and with JavaScript code that lets the user select part of the graph for a zoom-operation. Doesn't actually generate graph, only the HTML that links to the graph. 


 \textbf{action=view}
 Generate a single graph image. 


 
\subsection{OPTIONS}
\begin{description}
\item[--config=FILENAME] Loads the graph configuration file from FILENAME. If not specified, the file \motoserverconfig{hobbitgraph.cfg} is used. See the \emph{hobbitgraph.cfg(5)}
 for details about this file. 

 

\item[--env=FILENAME] Loads the environment settings defined in FILENAME before executing the CGI. 

 

\item[--rrddir=DIRECTORY] The top-level directory for the RRD files. If not specified, the directory given by the BBRRDS environment is used. 

 

\item[--save=FILENAME] Instead of returning the image via the CGI interface (i.e. on stdout), save the generated image to FILENAME. 

 

\item[--debug] Enable debugging output. 

 


\end{description}
\subsection{ENVIRONMENT}


 \textbf{QUERY\_STRING}
 Provided by the webserver CGI interface, this decides what graph to generate. 


 
\subsection{FILES}


 \textbf{hobbitgraph.cfg:}
 The configuration file determining how graphs are generated from RRD files. 


 
\subsection{SEE ALSO}
hobbitgraph.cfg(5), hobbit(7), rrdtool(1) 


%%%%%%%%%%%%%%%%%%%%%%%%%%%%%%%%%%%%%%%%%%%%%%%%%%%%%%%%%%%%%%%%%%%%%%%%%%%%%%
%
%%%%%%%%%%%%%%%%%%%%%%%%%%%%%%%%%%%%%%%%%%%%%%%%%%%%%%%%%%%%%%%%%%%%%%%%%%%%%%
\newpage
\section{BBCOMBOTEST.CFG}
\subsection{NAME}
 bbcombotest.cfg - Configuration of bbcombotest tool 

 
\subsection{SYNOPSIS}
\textbf{\motoserverconfig{bbcombotest.cfg}}


 
\subsection{DESCRIPTION}
\emph{bbcombotest(1)}
 uses it's own configuration file, \motoserverconfig{bbcombotest.cfg} Each line in this file defines a combined test. 

 
\subsection{FILE FORMAT}
 Each line of the file defines a new combined test. Blank lines and
 lines starting with a hash mark (\#) are treated as comments and
 ignored. 


  The configuration file uses the hostnames and testnames that are
  already used in your Hobbit bb-hosts file. These are then combined
  using normal logical operators - ``||'' for ``or'', ``\&\&'' for
  ``and'' etc. 



  A simple test - e.g. ``Web1.http'' - results in the value ``1'' if
  the ``http'' test for server ``Web1'' is green, yellow or clear. It
  yields the value ``0'' if it is red, purple or blue. 



  Apart from the logical operations, you can also do integer
  arithmetic and comparisons. E.g. the following is valid: 



  WebCluster.http = (Web1.http + Web2.http + Web3.http) $>$= 2 


  This test is green if two or more of the http tests for Web1, Web2 and Web3 are green. 


  The full range of operators are: 


  
+Add  
-Subtract  
*Multiply  
/Divide  
\%Modulo  
|Bit-wise''or''  
\&Bit-wise''and''  
||Logical''or''  
\&\&Logical''and''  
$>$Greaterthan  
$<$Lessthan  
$>$=Greaterthanorequal  
$<$=Lessthanorequal  
==Equal 


  There is currently no support for a ``not'' operator. If you need
  it, use the transcription ``(host.test == 0)'' instead of
  ``!host.test''. 



  NB: All operators have EQUAL PRECEDENCE. If you need something evaluated in a specific order, use parentheses to group the expressions together. 


  If the expression comes out as ``0'', the combined test goes red. If it comes out as non-zero, the combined test is green. 


  Note: If the expression involves hostnames with a character that is
  also an operator - e.g. if you have a host
  ``t1-router-newyork.foo.com'' with a dash in the hostname - then the
  operator-character must be escaped with a backslash '$\backslash$'
  in the expression, or it will be interpreted as an
  operator. E.g. like this: 



  
nyc.conn=(t1$\backslash$-router$\backslash$-nyc.conn||backup$\backslash$-router$\backslash$-nyc.conn) 


 


 
\subsection{EXAMPLE}
 WebCluster.http = (Web1.http || Web2.http)  
 AppSrvCluster.procs = (AppSrv1.conn \&\& AppSrv1.procs) || (AppSrv2.conn \&\& AppSrv2.procs)  
 Customer.cluster = WebCluster.http \&\& AppSrvCluster.procs  


  The first line defines a new test, with hostname ``WebCluster'' and
  the columnname ``http''. It will be green if the http test on either
  the ``Web1'' or the ``Web2'' server is green. 



  The second line defines a ``procs'' test for the ``AppSrvCluster''
  host. Each of the AppSrv1 and AppSrv2 hosts is checked for ``conn''
  (ping) and their ``procs'' test. On each host, both of these must be
  green, but the combined test is green if that condition is fulfilled
  on just one of the hosts. 



  The third line uses the two first tests to build a ``double
  combined'' test, defining a test that shows the overall health of
  the system. 

 
\subsection{FILES}
\textbf{\motoserverconfig{bbcombotest.cfg}}

 
\subsection{SEE ALSO}
bbcombotest(1) 


%%%%%%%%%%%%%%%%%%%%%%%%%%%%%%%%%%%%%%%%%%%%%%%%%%%%%%%%%%%%%%%%%%%%%%%%%%%%%%
%
%%%%%%%%%%%%%%%%%%%%%%%%%%%%%%%%%%%%%%%%%%%%%%%%%%%%%%%%%%%%%%%%%%%%%%%%%%%%%%
\newpage
\section{HOBBITGRAPH.CGI}


\subsection{NAME}
 hobbitgraph.cgi - CGI to generate Hobbit trend graphs \subsection{SYNOPSIS}
\textbf{hobbitgraph [options]}


 
\subsection{DESCRIPTION}
\textbf{hobbitgraph.cgi}
 is invoked as a CGI script via the hobbitgraph.sh CGI wrapper. 

  hobbitgraph.cgi is passed a QUERY\_STRING environment variable with the following parameters: 


 \textbf{host}
 Name of the host to generate a graph for 


 \textbf{service}
 Name of the service to generate a graph for 


 \textbf{disp}
 Display-name of the host, used on the generated graphs instead of hostname. 


 \textbf{graph}
 Can be ``hourly'', ``daily'', ``weekly'' or ``monthly'' to select the time period that the graph covers. 


 \textbf{first}
 Used to split multi-graphs into multiple graphs. This causes hobbitgraph.cgi to generate only the graphs starting with the ``first'th'' graph and continuing for ``count''. 


 \textbf{count}
 Number of graphs in a multi-graph. 


 \textbf{upper}
 Set the upper limit of the graph. See \emph{rrdgraph(1)}
 for a description of the ``-u'' option. 


 \textbf{lower}
 Set the lower limit of the graph. See \emph{rrdgraph(1)}
 for a description of the ``-l'' option. 


 \textbf{graph\_start}
 Set the starttime of the graph. This is used in zoom-mode. 


 \textbf{graph\_end}
 Set the end-time of the graph. This is used in zoom-mode. 


 \textbf{action=menu}
 Generate an HTML page with links to 4 graphs, representing the hourly, weekly, monthly and yearly graphs. Doesn't actually generate any graphs, only the HTML that links to the graphs. 


 \textbf{action=selzoom}
 Generate an HTML page with link to single graph, and with JavaScript code that lets the user select part of the graph for a zoom-operation. Doesn't actually generate graph, only the HTML that links to the graph. 


 \textbf{action=view}
 Generate a single graph image. 


 
\subsection{OPTIONS}
\begin{description}
\item[--config=FILENAME] Loads the graph configuration file from FILENAME. If not specified, the file \motoserverconfig{hobbitgraph.cfg} is used. See the \emph{hobbitgraph.cfg(5)}
 for details about this file. 

 

\item[--env=FILENAME] Loads the environment settings defined in FILENAME before executing the CGI. 

 

\item[--rrddir=DIRECTORY] The top-level directory for the RRD files. If not specified, the directory given by the BBRRDS environment is used. 

 

\item[--save=FILENAME] Instead of returning the image via the CGI interface (i.e. on stdout), save the generated image to FILENAME. 

 

\item[--debug] Enable debugging output. 

 


\end{description}
\subsection{ENVIRONMENT}


 \textbf{QUERY\_STRING}
 Provided by the webserver CGI interface, this decides what graph to generate. 


 
\subsection{FILES}


 \textbf{hobbitgraph.cfg:}
 The configuration file determining how graphs are generated from RRD files. 


 
\subsection{SEE ALSO}
hobbitgraph.cfg(5), hobbit(7), rrdtool(1) 

 
%%%%%%%%%%%%%%%%%%%%%%%%%%%%%%%%%%%%%%%%%%%%%%%%%%%%%%%%%%%%%%%%%%%%%%%%%%%%%%
%
%%%%%%%%%%%%%%%%%%%%%%%%%%%%%%%%%%%%%%%%%%%%%%%%%%%%%%%%%%%%%%%%%%%%%%%%%%%%%%
\newpage
\section{BB-SNAPSHOT.CGI}
\subsection{NAME}
 bb-snapshot.cgi - CGI program to rebuild the Hobbit webpages for a
 specific point in time. 
\subsection{SYNOPSIS}

\textbf{bb-snapshot.cgi}


 
\subsection{DESCRIPTION}
\textbf{bb-snapshot.cgi} is invoked as a CGI script via the
bb-snapshot.sh CGI wrapper. It rebuilds the Hobbit web pages to the
look they had at a particular point in time, based upon the historical
information logged about events. 


 bb-snapshot.cgi is passed a QUERY\_STRING environment variable with the following parameters: 


  
mon(Startmonthofthesnapshot)  
day(Startday-of-monthofthesnapshot)  
yr(Startyearofthesnapshot)  
hour(Starthourofthesnapshot)  
min(Startminuteofthesnapshot)  
sec(Startsecondofthesnapshot) 


  The ``month'' parameters must be specified as the three-letter
  english month name abbreviation: Jan, Feb, Mar ... 



  ``day'' must be in the range 1..31; ``yr'' must be specified
  including century (e.g. ``2003''). ``hour'' must be specified using
  a 24-hour clock. 



  All of the processing involved in generating the report is done by
  invoking \emph{bbgen(1)} with the proper ``--snapshot'' option. 


 
\subsection{OPTIONS}
\begin{description}
\item[--env=FILENAME] Load environment from FILENAME before executing the CGI. 

 

\item[bbgen-options] All options except ``--env'' are passed on to the \emph{bbgen(1)}
 program building the snapshot files. 

 


\end{description}


\subsection{ENVIRONMENT VARIABLES}
\begin{description}
\item[BBGENSNAPOPTS] bbgen options passed by default to the
  bb-snapshot.cgi script. This happens in the bb-snapshot.sh CGI
  wrapper script. 

\item[BBHOME] Home directory of the Hobbit server files 
\item[BBSNAP] Directory where generated snapshots are stored. This
  directory must be writable by the userid executing the CGI script,
  typically ``www'', ``apache'' or ``nobody''. Default:
  \$BBHOME/www/snap/ 

\item[BBSNAPURL] The URL prefix to use when accessing the reports via
  a browser. Default: \$BBWEB/snap 



\end{description}
\subsection{SEE ALSO}
bbgen(1), bb-hosts(5), hobbitserver.cfg(5) 


%%%%%%%%%%%%%%%%%%%%%%%%%%%%%%%%%%%%%%%%%%%%%%%%%%%%%%%%%%%%%%%%%%%%%%%%%%%%%%
%
%%%%%%%%%%%%%%%%%%%%%%%%%%%%%%%%%%%%%%%%%%%%%%%%%%%%%%%%%%%%%%%%%%%%%%%%%%%%%%
\newpage
\section{HOBBITD\_SAMPLE}
\subsection{NAME}
 hobbitd\_sample - example of a hobbitd worker module \subsection{SYNOPSIS}
\textbf{hobbitd\_channel --channel=status hobbitd\_sample [options]}

 
\subsection{DESCRIPTION}
 hobbitd\_sample is a worker module for hobbitd, and as such it is
 normally run via the \emph{hobbitd\_channel(8)} program. It receives
 messages from hobbitd via stdin, and simply displays these on
 stdout. It can be used with all types of hobbitd channels. 


  hobbitd\_sample is not designed to actually run, except as a
  demonstration. The purpose of this tool is to show how hobbitd
  worker modules can be implemented to handle different tasks that
  need to hook into the hobbitd processing. 


 
\subsection{OPTIONS}
\begin{description}
\item[--timeout=N] Read messages with a timeout of N seconds. 

 

\item[--debug] Enable debugging output. 

 


\end{description}
\subsection{SEE ALSO}
hobbitd\_channel(8), hobbitd(8), hobbit(7) 


%%%%%%%%%%%%%%%%%%%%%%%%%%%%%%%%%%%%%%%%%%%%%%%%%%%%%%%%%%%%%%%%%%%%%%%%%%%%%%
%
%%%%%%%%%%%%%%%%%%%%%%%%%%%%%%%%%%%%%%%%%%%%%%%%%%%%%%%%%%%%%%%%%%%%%%%%%%%%%%
\newpage
\section{BB-CSVINFO.CGI}
\subsection{NAME}
 bb-csvinfo.cgi - CGI program to show host information from a CSV file \subsection{SYNOPSIS}
\textbf{bb-csvinfo.cgi}


 
\subsection{DESCRIPTION}
\textbf{bb-csvinfo.cgi}
 is invoked as a CGI script via the bb-csvinfo.sh CGI wrapper. Based
 on the parameters it receives, it searches a comma- separated file
 for the matching host, and presents the information found as a
 table. 


  bb-csvinfo.cgi is passed a QUERY\_STRING environment variable with
  the following parameters: 



  
key(stringtosearchfor,typicallyhostname)  
column(columnnumbertosearch-default0)  
db(nameoftheCSVdatabasefilein\$BBHOME/etc/,defaulthostinfo.csv) 
delimiter(delimitercharacterforcolumns,defaultsemi-colon) 


  CSV files are easily created from e.g. spreadsheets, by exporting
  them in CSV format. You should have one host per line, with the
  first line containing the column headings. Despite their name, the
  default delimiter for CSV files is the semi-colon - if you need a
  different delimiter, invoke bb-csvinfo.cgi with the
  ``delimiter=$<$character$>$'' in the query string. 



 
\subsection{Example usage}
 This example shows how you can use the bb-csvinfo CGI. It assumes you
 have a CSV-formatted file with information about the hosts stored as
 \motoserverconfig{hostinfo.csv}, and the hostname is in the first column of
 the file. 


 \begin{description}
\item[Use with the bbgen --docurl] The --docurl option to \emph{bbgen(1)}
 sets up all of the hostnames on your Hobbit webpages to act as links
 to a CGI script. To invoke the bb-csvinfo CGI script, run bbgen with
 the option 


  
--docurl=/cgi-bin/bb-csvinfo.sh?db=hostinfo.csv\&key=\%s 



\end{description}

\subsection{SEE ALSO}
bb-hosts(5), hobbitserver.cfg(5), bbgen(1) 

 
%%%%%%%%%%%%%%%%%%%%%%%%%%%%%%%%%%%%%%%%%%%%%%%%%%%%%%%%%%%%%%%%%%%%%%%%%%%%%%
%
%%%%%%%%%%%%%%%%%%%%%%%%%%%%%%%%%%%%%%%%%%%%%%%%%%%%%%%%%%%%%%%%%%%%%%%%%%%%%%
\newpage
\section{HOBBIT-HOSTGRAPHS.CGI}
 hobbit-hostgraphs.cgi - CGI program to show multiple graphs 

\subsection{SYNOPSIS}
\textbf{hobbit-hostgraph.cgi}


\subsection{DESCRIPTION}
\textbf{hobbit-hostgraph.cgi} is invoked as a CGI script via the hobbit-hostgraph.sh CGI wrapper. 

  If no parameters are provided when invoked, it will present a form
  where the user can select a time period, one or more hosts, and a
  set of graphs. 



  The parameters selected by the user are passed to a second
  invocation of hobbit-hostgraph.cgi, and result in a webpage showing
  a list of graph images based on the trend data stored about the
  hosts. 



  If multiple graph-types are selected, hobbit-hostgraph.cgi will display a list of graphs, with one graph per type. 


  If multiple hosts are selected, hobbit-hostgraph.cgi will attempt to
  display a multi-host graph for each type where the graphs for all
  hosts are overlayed in a single image, allowing for easy comparison
  of the hosts. 



  The hostlist uses the PAGEPATH cookie provided by Hobbit webpages to
  select the list of hosts to present. Only the hosts visible on the
  page where hobbit-hostgraph.cgi is invoked from will be visible. 



  The resulting graph page can be bookmarked, but the bookmark also fixates the time period shown. 


 
\subsection{OPTIONS}
\begin{description}
\item[--env=FILENAME] Loads the environment defined in FILENAME before executing the CGI script. 

 


\end{description}
\subsection{BUGS}
 This utility is experimental. It may change in a future release of Hobbit. 

  It is possible for the user to select graphs which do not exist. This results in broken image links. 


  The set of graph-types is fixed in the server/web/hostgraphs\_form template and does not adjust to which graphs are available. 


  If the tool is invoked directly, all hosts defined in Hobbit will be listed. 


 
\subsection{SEE ALSO}
bb-hosts(5), hobbitserver.cfg(5) 

 

%%%%%%%%%%%%%%%%%%%%%%%%%%%%%%%%%%%%%%%%%%%%%%%%%%%%%%%%%%%%%%%%%%%%%%%%%%%%%%
%
%%%%%%%%%%%%%%%%%%%%%%%%%%%%%%%%%%%%%%%%%%%%%%%%%%%%%%%%%%%%%%%%%%%%%%%%%%%%%%
%\include{bb-datepage.cgi.1}

\newpage
\section{BB-DATEPAGE.CGI}

 bb-datepage.cgi - Hobbit CGI script to view pre-built reports by date \

\subsection{SYNOPSIS}

\textbf{bb-datepage.cgi?type={day,week,month} --url=URLPREFIX [options]}


 
\subsection{DESCRIPTION}
\textbf{bb-datepage.cgi}
 is invoked as a CGI script via the bb-datepage.sh CGI wrapper. 

  bb-datepage.cgi is passed a QUERY\_STRING environment variable with
  the type of time-selection that is desired: Either ``day'', ``week''
  or ``month'' can be requested. It will then generate a web form with
  appropriate day/week/month selection boxes, and based on the users'
  selection a resulting url is built from the URLPREFIX and the time
  selection. The browser is then redirected to this URL. 



  The URL is constructed from the URLPREFIX, the type-parameter, the
  value of the ``pagepath'' or ``host'' cookie, and the users'
  selection as follows: 



 \begin{description}
\item[type=day] The final URL is \textbf{URLPREFIX/daily/YEAR/MONTH/DAY/PAGEPATH}. 

 

\item[type=week] The final URL is \textbf{URLPREFIX/weekly/YEAR/WEEK/PAGEPATH}
. 

 

\item[type=month] The final URL is \textbf{URLPREFIX/monthly/YEAR/MONTH/PAGEPATH}
. 

  YEAR is the full year (4 digits, including century). MONTH is the
  two-digit number of the month (01..12). DAY is the number of the day
  in the month (01..31). WEEK is the ISO 8601:1988 week-number
  (01..53). PAGEPATH is the current value of the ``pagepath'' cookie
  if set; if it is not set but the ``host'' cookie is set, then this
  host is looked up in the bb-hosts file and the page where this host
  is found is used for PAGEPATH. These two cookies are set by the
  default web-header templates supplied with Hobbit. 



\end{description}

\subsection{OPTIONS}
\begin{description}
\item[--url=URLPREFIX] This specifies the initial part of the final URL. This option is required. 

 

\item[--hffile=FILENAME] Specifies the template files (from \$BBHOME/web/) to use. The default is ``--hffile=report''. 

 

\item[--color=COLOR] Sets the background color of the generated webpage. The default is blue. 

 

\item[--env=FILENAME] Loads the environment defined in FILENAME before executing the CGI script. 

 

\item[--debug] Enables debugging output. 

 

\item[\$BBHOME/web/report\_form\_daily] HTML form template for the date selection form when type=daily. 

 

\item[\$BBHOME/web/report\_form\_weekly] HTML form template for the date selection form when type=weekly. 

 

\item[\$BBHOME/web/report\_form\_monthly] HTML form template for the date selection form when type=monthly. 

 

\item[\$BBHOME/web/report\_header] HTML header file for the generated web page 

 

\item[\$BBHOME/web/report\_footer] HTML footer file for the generated web page 

 


\end{description}
\subsection{ENVIRONMENT VARIABLES}
\begin{description}
\item[BBHOME] Used to locate the template files for the generated web pages. 

 

\item[QUERY\_STRING] Contains the parameters for the CGI script. 

 


\end{description}
\subsection{SEE ALSO}
bbgen(1), bb-hosts(5), hobbitserver.cfg(5) 

 
%%%%%%%%%%%%%%%%%%%%%%%%%%%%%%%%%%%%%%%%%%%%%%%%%%%%%%%%%%%%%%%%%%%%%%%%%%%%%%
%
%%%%%%%%%%%%%%%%%%%%%%%%%%%%%%%%%%%%%%%%%%%%%%%%%%%%%%%%%%%%%%%%%%%%%%%%%%%%%%
%\include{bbdigest.1}
\newpage
\section{BBDIGEST}
 bbdigest - calculate message digests

\subsection{SYNOPSIS}
\textbf{bbdigest md5|sha1|rmd160 [filename]}


 
\subsection{DESCRIPTION}
\emph{bbdigest(1)}
 is a utility to calculate message digests for a file or document. It
 is used when defining HTTP- or FTP-based content checks, where
 \emph{bbtest-net(1)} checks that a URL returns a specific document;
 instead of having to compare the entire document, the comparison is
 done against a pre-computed message digest value using the MD5,
 RIPEMD160, SHA1 or any of the SHA2 (SHA-512, SHA-256, SHA-384,
 SHA-224) message digest algorithms. 


  The optional \textbf{filename} parameter is the input file whose
  message digest should be calculated; if no filename is given, the
  data is read from standard input. 



  bbdigest outputs a string containing the digest algorithm and the
  computed message digest. This is in a format suitable for use in the
  \emph{bb-hosts(5)}

 definition of a content check. 


 
\subsection{EXAMPLE}


\$bbdigestmd5index.html  md5:88b81b110a85c83db56a939caa2e2cf6 
  
\$curl-s\url{http://www.foo.com/}|bbdigestsha1  sha1:e5c69784cb971680e2c7380138e04021a20a45a2 

 
\subsection{SEE ALSO}
bbtest-net(1), bb-hosts(5) 

 


%%%%%%%%%%%%%%%%%%%%%%%%%%%%%%%%%%%%%%%%%%%%%%%%%%%%%%%%%%%%%%%%%%%%%%%%%%%%%%
%
%%%%%%%%%%%%%%%%%%%%%%%%%%%%%%%%%%%%%%%%%%%%%%%%%%%%%%%%%%%%%%%%%%%%%%%%%%%%%%
\newpage
\section{HOBBITLAUNCH.CFG}

 hobbitlaunch.cfg - Task definitions for the hobbitlaunch utility 

 
\subsection{SYNOPSIS}
\textbf{\motoserverconfig{hobbitlaunch.cfg}}


 
\subsection{DESCRIPTION}
 The hobbitlaunch.cfg file holds the list of tasks that hobbitlaunch
 runs to perform all of the tasks needed by the Hobbit monitor. 


 
\subsection{FILE FORMAT}
 A task is defined by a \textbf{key}
, a \textbf{command}
, and optionally also \textbf{interval}
, \textbf{environment}
, and \textbf{logfile}
. 

  Blank lines and lines starting with a hash mark (\#) are treated as
  comments and ignored. Long lines can be broken up by putting a
  backslash at the end of the line and continuing the entry on the
  next line. 



  An entry looks like this: 


  
[hobbitd]  
 
ENVFILE/usr/local/hobbit/server/etc/hobbitserver.cfg  
 
CMD/usr/local/hobbit/server/bin/hobbitd 


  
[updateweb]  
 
ENVFILE/usr/local/hobbit/server/etc/hobbitserver.cfg  
 
CMD/usr/local/hobbit/server/bin/bbgen  
 
NEEDShobbitd  
 
GROUPwebupdates  
 
INTERVAL5m  
 
LOGFILE/var/log/hobbit/updateweb.log 


  The \textbf{key} is enclosed in angle brackets, and must be unique
  for each task. You can choose your key-names as you like, they are
  only used internally in hobbitlaunch to identify each task. 



  The \textbf{command}
 is defined by the CMD keyword. This is the full command including any
 options you want to use for this task. This is required for all
 tasks. 



  The \textbf{DISABLED}
 keyword means that this command is disabled. hobbitlaunch will not
 start this task. It is recommended that you use this to disable
 standard tasks, instead of removing them or commenting them
 out. Upgrades to Hobbit will add standard tasks back into the file,
 so unless you have them listed as DISABLED then tasks may re-appear
 unexpectedly after an upgrade. 



  The \textbf{ONHOST}
 keyword tells hobbitlaunch that this task should only run on specific
 hosts. After the ONHOST keyword, you must provide a ``regular
 expression''; if the hostname where hobbitlaunch runs matches this
 expression, then the task will run. If it doesn't match, then the
 task is treated as if it were DISABLED. 



  The \textbf{NEEDS}
 instructs hobbitlaunch not to run this task unless the task defined
 by the NEEDS keyword is already running. This is used e.g. to delay
 the start of some application until the needed daemons have been
 started. The task that must be running is defined by its \textbf{key}

. 


  The \textbf{GROUP}
 keyword can be used to limit the number of tasks that may run
 simultaneously. E.g. if you are generating multiple pagesets of
 webpages, you dont want them to run at the same time. Putting them
 into a GROUP will cause hobbitlaunch to delay the start of new tasks,
 so that only one task will run per group. You can change the limit by
 defining the group before the tasks, with a ``GROUP groupname
 maxtasks'' line. 



  The \textbf{INTERVAL}
 keyword defines how often this command is executed. The example shows
 a command that runs every 5 minutes. If no interval is given, the
 task is only run once - this is useful for tasks that run continually
 as daemons - although if the task stops for some reason, then
 hobbitlaunch will attempt to restart it. Intervals can be specified
 in seconds (if you just put a number there), or in minutes (5m),
 hours (2h), or days (1d). 



  The \textbf{ENVFILE}
 setting points to a file with definitions of environment
 variables. Before running the task, hobbitlaunch will setup all of
 the environment variables listed in this file. Since this is a
 per-task setting, you can use the same hobbitlaunch instance to run
 e.g. both the server- and client-side Hobbit tasks. If this option is
 not present, then the environment defined to hobbitlaunch is used. 



  The \textbf{ENVAREA}
 setting modifies which environment variables are loaded, by picking
 up the ones that are defined for this specific ``area''. See
 \emph{hobbitserver.cfg(5)} for information about environment areas. 


  The \textbf{LOGFILE}
 setting defines a logfile for the task. hobbitlaunch will start the
 task with stdout and stderr redirected to this file. If this option
 is not present, then the output goes to the same location as the
 hobbitlaunch output. 



 
\subsection{SEE ALSO}
hobbitlaunch(8), hobbitd(8), hobbit(7) 

%%%%%%%%%%%%%%%%%%%%%%%%%%%%%%%%%%%%%%%%%%%%%%%%%%%%%%%%%%%%%%%%%%%%%%%%%%%%%%
%
%%%%%%%%%%%%%%%%%%%%%%%%%%%%%%%%%%%%%%%%%%%%%%%%%%%%%%%%%%%%%%%%%%%%%%%%%%%%%%
\newpage
\section{BB-EVENTLOG.CGI}

 bb-eventlog.cgi - CGI program to report the Hobbit eventlog \subsection{SYNOPSIS}
\textbf{bb-eventlog.cgi}


 

\subsection{DESCRIPTION}
\textbf{bb-eventlog.cgi}
 is invoked as a CGI script via the bb-eventlog.sh CGI wrapper. Based
 on the parameters it receives, it generates the Hobbit event log for
 a period. This log shows all status changes that have occurred for
 all hosts and services. 


  bb-eventlog.cgi is passed a QUERY\_STRING environment variable with the following parameters: 


  
MAXTIME(maximumminutestogobackinthelog)  
MAXCOUNT(maximumnumberofeventstoreport) 


 
\subsection{OPTIONS}
\begin{description}
\item[--top=N] Shows a summary of the top-N changing hosts and services. 

 

\item[--env=FILENAME] Loads the environment defined in FILENAME before executing the CGI script. 

\end{description}
\subsection{SEE ALSO}
bb-hosts(5), hobbitserver.cfg(5) 

%%%%%%%%%%%%%%%%%%%%%%%%%%%%%%%%%%%%%%%%%%%%%%%%%%%%%%%%%%%%%%%%%%%%%%%%%%%%%%
%
%%%%%%%%%%%%%%%%%%%%%%%%%%%%%%%%%%%%%%%%%%%%%%%%%%%%%%%%%%%%%%%%%%%%%%%%%%%%%%
%\include{hobbit-mailack.8}
\newpage
\section{HOBBIT-MAILACK}

 hobbit-mailack - permit acknowledging alerts via e-mail \subsection{SYNOPSIS}
\textbf{hobbit-mailack --env=FILENAME [--debug]}


 
\subsection{DESCRIPTION}
 hobbit-mailack normally runs as an input mail-filter for the hobbit user, e.g. by being called from the hobbit users' \emph{procmailrc(5)}
 file. hobbit-mailack recognizes e-mails that are replies to \emph{hobbitd\_alert(8)}
 mail alerts, and converts the reply mail into an acknowledge message that is sent to the Hobbit system. This permits an administrator to acknowledge an alert via e-mail. 

 
\subsection{ADDING INFORMATION TO THE REPLY MAIL}
 By default, an acknowledgment is valid for 1 hour. If you know in advance that solving the problem is going to take longer, you can change this by adding \textbf{delay=DURATION}
 to the subject of your mail reply or on a line in the reply message. Duration is in minutes, unless you add a trailing 'h' (for 'hours'), 'd' (for 'days') or 'w' (for 'weeks'). 

  You can also include a message that will show up on the status-page together with the acknowledgment, e.g. to provide an explanation for the issue or some other information to the users. You can either put it at the end of the subject line as \textbf{msg=Some random text}
, or you can just enter it in the e-mail as the first non-blank line of text in the mail (a ``delay=N'' line is ignored when looking for the message text). 


 
\subsection{USE WITH PROCMAIL}
 To setup hobbit-mailack, create a \textbf{.procmailrc}
 file in the hobbit-users home-directory with the following contents: \begin{description}
\item[]\begin{verbatim}

DEFAULT=$HOME/Mailbox
LOGFILE=$HOME/procmail.log
:0
| $HOME/server/bin/hobbit-mailack --env=$HOME/server/etc/hobbitserver.cfg

\end{verbatim}


 


\end{description}
\subsection{USE WITH QMAIL}
 If you are using Qmail to deliver mail locally, you can run hobbit-mailack directly from a \textbf{.qmail}
 file. Setup the hobbit-users .qmail file like this: \begin{description}
\item[]\begin{verbatim}

| $HOME/server/bin/hobbit-mailack --env=$HOME/server/etc/hobbitserver.cfg

\end{verbatim}


 


\end{description}
\subsection{OPTIONS}
\begin{description}
\item[--env=FILENAME] Load environment from FILENAME, usually hobbitserver.cfg. 

 

\item[--debug] Dont send a message to hobbitd, but dump the message to stdout. 

 


\end{description}
\subsection{SEE ALSO}
hobbitd\_alert(8), hobbitd(8), hobbit(7) 

 
%%%%%%%%%%%%%%%%%%%%%%%%%%%%%%%%%%%%%%%%%%%%%%%%%%%%%%%%%%%%%%%%%%%%%%%%%%%%%%
%
%%%%%%%%%%%%%%%%%%%%%%%%%%%%%%%%%%%%%%%%%%%%%%%%%%%%%%%%%%%%%%%%%%%%%%%%%%%%%%
\newpage
\section{BB-FINDHOST.CGI}

 bb-findhost.cgi - Hobbit CGI script to find hosts

\subsection{SYNOPSIS}
\textbf{bb-findhost.cgi?host=REGEX}


 
\subsection{DESCRIPTION}
\textbf{bb-findhost.cgi}
 is invoked as a CGI script via the bb-findhost.sh CGI wrapper. 

  bb-findhost.cgi is passed a QUERY\_STRING environment variable with
  the ``host=REGEX'' parameter. The REGEX is a Posix regular
  expression (see \emph{regex(7)} ) describing the hostnames to look
  for. A trailing wildcard is assumed on all hostnames -
  e.g. requesting the hostname ``www'' will match any host whose name
  begins with ``www''. 

  It then produces a single web page, listing all of the hosts that
  matched any of the hostnames, with links to the Hobbit webpages
  where they are located. 

  The output page lists hosts in the order they appear in the
  \emph{bb-hosts(5)} file. 



  A sample web page implementing the search facility is included with
  bbgen, you access it via the URL /bb/help/bb-findhost.html. 



 
\subsection{OPTIONS}
\begin{description}
\item[--env=FILENAME] Loads the environment from FILENAME before executing the CGI. 

 


\end{description}
\subsection{FILES}
\begin{description}
\item[\$BBHOME/web/findhost\_header] HTML header file for the generated web page 

 

\item[\$BBHOME/web/findhost\_footer] HTML footer file for the generated web page 

 

\item[\$BBHOME/web/findhost\_form] Query form displayed when
  bb-findhost.cgi is called with no parameters. 


\end{description}
\subsection{ENVIRONMENT VARIABLES}
\begin{description}
\item[BBHOSTS] bb-findhost.cgi uses the BBHOSTS environment variable
  to find the bb-hosts file listing all known hosts and their page
  locations. 


\item[BBHOME] Used to locate the template files for the generated web pages. 


\end{description}
\subsection{SEE ALSO}
bbgen(1), bb-hosts(5), hobbitserver.cfg(5) 

 
%%%%%%%%%%%%%%%%%%%%%%%%%%%%%%%%%%%%%%%%%%%%%%%%%%%%%%%%%%%%%%%%%%%%%%%%%%%%%%
%
%%%%%%%%%%%%%%%%%%%%%%%%%%%%%%%%%%%%%%%%%%%%%%%%%%%%%%%%%%%%%%%%%%%%%%%%%%%%%%
\newpage
\section{HOBBITCLIENT.CFG}

 hobbitclient.cfg - Hobbit client environment variables 

 

\subsection{DESCRIPTION}
 Hobbit programs use multiple environment variables beside the normal
 set of variables. For the Hobbit client, the environment definitions
 are stored in the ~hobbit/client/etc/hobbitclient.cfg file. Each line
 in this file is of the form \textbf{NAME=VALUE} and defines one
 environment variable NAME with the value VALUE. 


 
\subsection{SETTINGS}

\begin{description}

\item[BBDISP] The IP-address used to contact the Hobbit
  server. Default: Chosen when the Hobbit client was compiled. 


 

\item[BBDISPLAYS] List of IP-adresses of Hobbit servers. Data will be
  sent to all of the servers listed here. This setting is only used if
  BBDISP=0.0.0.0. 


 

\item[BBPORT] The portnumber for used to contact the Hobbit server. Default: 1984. 

 

\item[BBHOME] The Hobbit client top-level directory. Default: The
  \$HOBBITCLIENTHOME setting inherited from the ``runclient.sh''
  script which starts the Hobbit client. 


 

\item[BBCLIENTLOGS] The directory for the Hobbit clients' own logfiles. Default: \$BBHOME/logs 

 

\item[BBTMP] Directory used for temporary files. Default: \$BBHOME/tmp/ 

 

\item[BB] Full path to the \emph{bb(1)}
 client program. Default: \$BBHOME/bin/bb. 

 

\item[Commands] Many extension scripts expect a series of environment
  variables to point at various system utilities. These are included
  in the file when the client is built. 



\end{description}

\subsection{INHERITED SETTINGS}
 Some environment variables are inherited from the ``runclient.sh'' script which launches the Hobbit client: 

 \begin{description}
\item[MACHINEDOTS] The hostname of the local system. Default: Taken from ``uname -n''. 

 

\item[MACHINE] The hostname of the local system, with dots replaced by
  commas. For compatibility with Big Brother extension scripts. 


 

\item[BBOSTYPE] The operating system of the local system, in lowercase. Default: taken from ``uname -s''. 

 

\item[HOBBITCLIENTHOME] The top-level directory for the Hobbit
  client. Default: The location of the ``runclient.sh'' script. 



\end{description}

\subsection{SEE ALSO}
hobbit(7) 
 
