\chapter{How to setup custom graphs}
\label{chap:How to setup custom graphs}

\section{How to setup custom graphs}
 This document walks you through the setup of custom graphs in your
 Hobbit installation. Although Hobbit comes with pre-defined setups
 for a lot of common types of graphs, it is also extensible allowing
 you to add your own tests. For many kinds of tests, it is nice to
 view them over a period of time in a graph - this document tells you
 how to do that. \subsection{Make a script to collect the data}




 First create your test data. Typically, this is an extension script
 that sends in some data to Hobbit, using a \textbf{status}

 or \textbf{data}
 command. If you use \textbf{status}
, it will show up as a separate column on the display, with a
 green/yellow/red color that can trigger alerts. If you use
 \textbf{data}

, Hobbit just collects the data into a graph - you must go to the

 \textbf{trends}
 column to see the graph. For this example, we'll use \textbf{status}
.


 So we create an extension script. Here is an example script; it picks
 two numbers out of the Linux kernel's memory statistics, and reports
 these to hobbit. 

\begin{verbatim}


	#!/bin/sh

	cat /proc/slabinfo | \
	   egrep "^dentry_cache|^inode_cache" | \
	      awk '{print $1 " : " $3*$4}' >/tmp/slab.txt

	$BB $BBDISP "status $MACHINE.slab green `date`

	`cat /tmp/slab.txt`
	"

	exit 0

\end{verbatim}

\subsection{Get hobbitlaunch to run the script}


 Save this script in ~hobbit/client/ext/slab, and add a section to the
 ~hobbit/client/etc/clientlaunch.cfg to run it every 5 minutes:
 \begin{verbatim}

	[slabinfo]
        	ENVFILE /usr/lib/hobbit/client/etc/hobbitclient.cfg
	        CMD /usr/lib/hobbit/client/ext/slab
		INTERVAL 5m

\end{verbatim}

 (On the Hobbit server itself, you must add this to the file ~hobbit/server/etc/hobbitlaunch.cfg) 
\subsection{Check that the script data arrives in Hobbit}


 After a few minutes, a \textbf{slab}
 column should appear on your Hobbit view of this host, with the data it reports. The output looks like this: \begin{verbatim}

	Sun Nov 20 09:03:44 CET 2005

	inode_cache : 330624
	dentry_cache : 40891068

\end{verbatim}

\subsection{Arrange for the data to be collected into an RRD file}


 This is obviously a name-colon-value formatted report, so we'll use the NCV module in Hobbit to handle it. Hobbit will find two datasets here: The first will be called \textbf{inodecache}
, and the second \textbf{dentrycache}
 (note that Hobbit strips off any part of the name that is not a letter or a number; Hobbit also limits the length of the dataset name to 19 letters max. since RRD will not handle longer names). To enable this, on the Hobbit server edit the ~hobbit/server/etc/hobbitserver.cfg file. The TEST2RRD setting defines how Hobbit tests (status columns) map to RRD datafiles. So you add the new test to this setting, by adding \textbf{slab=ncv}
 at the end: \begin{verbatim}

TEST2RRD="cpu=la,disk,<...lots more stuff...>,hobbitd,mysql=ncv,slab=ncv"

\end{verbatim}



 \textbf{slab}
 is the status column name, and \textbf{=ncv}
 is a token that tells Hobbit to send these data through the built-in NCV module.


 By default, the Hobbit NCV module expects data to be some sort of
 counter, e.g. number of bytes sent over a network - it uses the RRD
 \textbf{DERIVE}

 datatype by default, which is for data that is continuously
 increasing in value. Some data are not like that - the data in our
 test script is not - and for those data you'll have to make an extra
 setting to tell Hobbit what RRD data type to use. The RRDtool
 rrdcreate(1) man-page has a detailed description of the various RRD
 datatypes. It is available online at
 \url{http://people.ee.ethz.ch/~oetiker/webtools/rrdtool/doc/rrdcreate.en.html}



 Our test script provides data that goes up and down in value (it is
 the number of bytes of memory used for a Linux kernel bufffer), and
 for that kind of data we'll use the RRD \textbf{GAUGE}

 datatype. So we add an extra setting to hobbitserver.cfg:
\begin{verbatim}

	NCV_slab="inodecache:GAUGE,dentrycache:GAUGE"

\end{verbatim}


 This tells the hobbitd\_rrd module that it should create an RRD file
 with two datasets of type GAUGE instead of the default (DERIVE). The
 setting must be named NCV\_$<$columnname$>$.



 The hobbitserver.cfg file is not reloaded automatically, so you must
 restart Hobbit after making these changes. Or at least, kill the
 hobbitd\_rrd processes (there are usually two) - hobbitlaunch will
 automatically restart them, and they will then pick up the new
 settings.

\subsection{Check that the RRD collects data}


 The next time the \textbf{slab}
 status is updated, Hobbit will begin to collect the data. You can
 check this by looking for the \textbf{slab.rrd}

 file in the ~hobbit/data/rrd/HOSTNAME/ directory. If you want to
 check the data it collects, the \textbf{rrdtool dump
 ~hobbit/data/rrd/HOSTNAME/slab.rrd}

 will tell you what it got: \begin{verbatim}

	<!-- Round Robin Database Dump -->
	<rrd>
		<version> 0001 </version>
		<step> 300 </step> <!-- Seconds -->
		<lastupdate> 1132474725 </lastupdate> <!-- 2005-11-20 09:18:45 CET -->

		<ds>
			<name> inodecache </name>
RRD datatype------>	<type> GAUGE </type>
			<minimal_heartbeat> 600 </minimal_heartbeat>
			<min> 0.0000000000e+00 </min>
			<max> NaN </max>

			<!-- PDP Status -->
current value----->	<last_ds> 330624 </last_ds>
			<value> 0.0000000000e+00 </value>
			<unknown_sec> 0 </unknown_sec>
		</ds>

\end{verbatim}



 If you go and look at the status page for the \textbf{slab}
 column, you should not see any graph yet, but a link to \textbf{hobbit graph ncv:slab}
. One final step is missing.

\subsection{Setup a graph definition}


 The final step is to tell Hobbit how to create a graph from the data in the RRD file. This is done in the ~hobbit/server/etc/hobbitgraph.cfg file. \begin{verbatim}

	[slab]
		TITLE Slab info
		YAXIS Bytes
		DEF:inode=slab.rrd:inodecache:AVERAGE
		DEF:dentry=slab.rrd:dentrycache:AVERAGE
		LINE2:inode#00CCCC:Inode cache
		LINE2:dentry#FF0000:Dentry cache
		COMMENT:\n
		GPRINT:inode:LAST:Inode cache \: %5.1lf%s (cur)
		GPRINT:inode:MAX: \: %5.1lf%s (max)
		GPRINT:inode:MIN: \: %5.1lf%s (min)
		GPRINT:inode:AVERAGE: \: %5.1lf%s (avg)\n
		GPRINT:dentry:LAST:Dentry cache\: %5.1lf%s (cur)
		GPRINT:dentry:MAX: \: %5.1lf%s (max)
		GPRINT:dentry:MIN: \: %5.1lf%s (min)
		GPRINT:dentry:AVERAGE: \: %5.1lf%s (avg)\n

\end{verbatim}



 \textbf{[slab]}
 is the name of this graph, and it must match the name of your status column if you want the graph to appear together with the status. The TITLE and YAXIS settings define the graph title and the legend on the Y-axis. The rest are definitions for the rrdgraph(1) tool - you should read the RRDtool docs if you want to know in detail how it works. For now, all you need to know is that you must pick out the data you want from the RRD file with a \textbf{DEF}
 line, like \begin{verbatim}

		DEF:inode=slab.rrd:inodecache:AVERAGE

\end{verbatim}
 which gives you an ``inode'' definition that has the value from the \textbf{inodecache}
 dataset in the slab.rrd file. This is then used to draw a line on the graph: \begin{verbatim}

		LINE2:inode#00CCCC:Inode cache

\end{verbatim}
 The line gets the color \textbf{\#00CCCC}
 (red-green-blue), which is a light greenish-blue color. Note that you can have several lines in one graph, if it makes sense to compare them. You can also use other types of visual effects, e.g. stack values on top of each other (like the \textbf{vmstat}
 graphs do) - this is described in the rrdgraph man-page. An online version is at  \url{http://people.ee.ethz.ch/~oetiker/webtools/rrdtool/doc/rrdgraph.en.html}.


 The GPRINT lines at the end of the graph definition also uses the \textbf{inode}
 value to print a summary line showing the current, maximum, minimum and average values from the data that has been collected.


 Once you have added this section to hobbitgraph.cfg, refresh the status page in your browser, and the graph should show up.
\subsection{Add the graph to the collection of graphs on the trends column}


 If you want the graph included with the other graphs on the \textbf{trends}
 column, you must add it to the GRAPHS setting in the ~hobbit/server/etc/hobbitserver.cfg file.
\begin{verbatim}

	GRAPHS="la,disk,<... lots more ...>,bbproxy,hobbitd,slab"

\end{verbatim}
 Save the file, and when you click on the \textbf{trends}
 column you should see the slab graph at the bottom of the page. \subsection{Common problems and pitfalls}
\subsubsection{If your graph nearly always shows 0}


 You probably used the wrong RRD datatype for your data - see step 4. By default, the RRD file expects data that is increasing constantly; if you are tracking some data that just varies up and down, you must use the RRD GAUGE datatype. Note that when you change the RRD datatype, you must delete any existing RRD files - the RRD datatype is defined when the RRD file is created, and cannot be changed on the fly.
\subsubsection{No graph on the status page, but OK on the trends page}


 Make sure you have \textbf{ncv}
 listed in the GRAPHS setting in hobbitserver.cfg. (Don't ask why - just take my word that it must be there).

